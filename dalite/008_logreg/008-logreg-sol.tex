\documentclass[letterpaper,9pt,twoside,printwatermark=false]{pinp}

%% Some pieces required from the pandoc template
\providecommand{\tightlist}{%
  \setlength{\itemsep}{0pt}\setlength{\parskip}{0pt}}

% Use the lineno option to display guide line numbers if required.
% Note that the use of elements such as single-column equations
% may affect the guide line number alignment.

\usepackage[T1]{fontenc}
\usepackage[utf8]{inputenc}

% The geometry package layout settings need to be set here...
\geometry{layoutsize={0.95588\paperwidth,0.98864\paperheight},%
          layouthoffset=0.02206\paperwidth,%
		  layoutvoffset=0.00568\paperheight}

\definecolor{pinpblue}{HTML}{185FAF}  % imagecolorpicker on blue for new R logo
\definecolor{pnasbluetext}{RGB}{101,0,0} %



\title{DALITE Q8 - Logistic Regression. Solutions.}

\author[a]{EPIB607 - Inferential Statistics}

  \affil[a]{Fall 2018, McGill University}

\setcounter{secnumdepth}{5}

% Please give the surname of the lead author for the running footer
\leadauthor{Bhatnagar and Hanley}

% Keywords are not mandatory, but authors are strongly encouraged to provide them. If provided, please include two to five keywords, separated by the pipe symbol, e.g:
 \keywords{  Two sample proportions |  Odds ratio |  Logit transformation |  Odds |  Log odds |  Risk calculator  }  

\begin{abstract}
This DALITE quiz will cover logistic regression.
\end{abstract}

\dates{This version was compiled on \today}
\doi{\url{https://sahirbhatnagar.com/EPIB607/}}

\pinpfootercontents{DALITE Q8 due November 18, 2018 by 11:59pm}

\begin{document}

% Optional adjustment to line up main text (after abstract) of first page with line numbers, when using both lineno and twocolumn options.
% You should only change this length when you've finalised the article contents.
\verticaladjustment{-2pt}

\maketitle
\thispagestyle{firststyle}
\ifthenelse{\boolean{shortarticle}}{\ifthenelse{\boolean{singlecolumn}}{\abscontentformatted}{\abscontent}}{}

% If your first paragraph (i.e. with the \dropcap) contains a list environment (quote, quotation, theorem, definition, enumerate, itemize...), the line after the list may have some extra indentation. If this is the case, add \parshape=0 to the end of the list environment.


\section*{FINRISK Calculator}\label{finrisk-calculator}
\addcontentsline{toc}{section}{FINRISK Calculator}

There are now several risk calculators derived from epidemiological
studies, such as the Framingham Heart Study. See, e.g.,
\url{https://www.uptodate.com/home/medical-calculators} and
\url{https://qxmd.com/calculate/}.

Many of the online ones for 10 year risks are based on the proportional
hazards model. One risk equation that is based on logistic regression
(see Rothman chapter 12, 2012 or Chapter 10, 2002) is that described in
the article Predicting Coronary Heart Disease and Stroke The FINRISK
Calculator by Erkki Vartiainen et al. GLOBAL HEART, VOL. 11, NO. 2, 2016
June 2016: 213-216 {[}article in myCourses{]}

By applying the equation in Table 2 of the article for the 10 year risk
of coronary heart disease for 55 year old females who don't smoke, but
whose cholesterol is 6, systolic blood pressure is 140, LDL is 0.5, are
NOT diabetic, and do not have a family history of MI, one obtains 9.3\%.

The same equation applied to females with the same profile, except that
they ARE diabetic, yields a 10 year risk of 22.3\%.

Thus, with respect to the No Diabetes (ref. category) vs.~Diabetes
(index category) contrast, say if the following statement is TRUE or
FALSE and explain why in your rationale:

\section{The risk difference is 13\%}\label{the-risk-difference-is-13}

\begin{enumerate}
\def\labelenumi{\alph{enumi}.}
\tightlist
\item
  \textbf{TRUE (Correct)}
\item
  FALSE
\end{enumerate}

\subsection{Correct rationales}\label{correct-rationales}

\begin{itemize}
\tightlist
\item
  When we want to calculate the risk difference we have to subtract the
  two risks from either group so in this case that would be 22.3\% -
  9.3\%= 13\%
\end{itemize}

\subsection{Incorrect rationales}\label{incorrect-rationales}

\section{The odds ratio is 2.4}\label{the-odds-ratio-is-2.4}

\begin{enumerate}
\def\labelenumi{\alph{enumi}.}
\tightlist
\item
  TRUE
\item
  \textbf{FALSE (Correct)}
\end{enumerate}

\subsection{Correct rationales}\label{correct-rationales-1}

\begin{itemize}
\tightlist
\item
  Estimate for the risk ratio rather than the odds.
\item
  2.4 is the risk ratio (when we divide 22.3/9.3) This is not the odds
  ratio
\end{itemize}

\subsection{Incorrect rationales}\label{incorrect-rationales-1}

\begin{itemize}
\tightlist
\item
  You divide the risk of the exposed over the risk of the unexposed to
  get the odds ratio. In this case 22.3/9.3 = 2.40, answer A) True.
\end{itemize}

\section{The risk ratio is 2.8}\label{the-risk-ratio-is-2.8}

\begin{enumerate}
\def\labelenumi{\alph{enumi}.}
\tightlist
\item
  TRUE
\item
  \textbf{FALSE (Correct)}
\end{enumerate}

\subsection{Correct rationales}\label{correct-rationales-2}

\begin{itemize}
\tightlist
\item
  The odds ratio is 2.8 ((0.223)/(1-0.223)/(0.093/(1-0.093)
\end{itemize}

\subsection{Incorrect rationales}\label{incorrect-rationales-2}

\begin{itemize}
\item
  \begin{enumerate}
  \def\labelenumi{(\arabic{enumi})}
  \tightlist
  \item
    Find odds first Odds (no diabetes) = 0.093 / (1 - 0.093) = 0.1025
    Odds (diabetes) = 0.223 / (1 - 0.223) = 0.2870 (2) Exponentiate the
    odds No diabetes: exp(0.1025) = 1.107937 Diabetes: exp(0.2870) =
    1.332424 Odds ratio = 1.352424 / 1.107937 = 1.20 Therefore it is
    false.
  \end{enumerate}
\end{itemize}

\section{The two risks can be connected via the equation: risk(\%) =
22.3 - 13 *
Diabetes}\label{the-two-risks-can-be-connected-via-the-equation-risk-22.3---13-diabetes}

\begin{enumerate}
\def\labelenumi{\alph{enumi}.}
\tightlist
\item
  TRUE
\item
  \textbf{FALSE (Correct)}
\end{enumerate}

\subsection{Correct rationales}\label{correct-rationales-3}

\begin{itemize}
\tightlist
\item
  Since no diabetes is the reference category (X= 0); which is risk(\%)
  = 9.3 + 13*Diabetes
\end{itemize}

\subsection{Incorrect rationales}\label{incorrect-rationales-3}

\section{The two risks can be connected via the equation: log{[}risk{]}
= -2.38 + 0.87 *
Diabetes}\label{the-two-risks-can-be-connected-via-the-equation-logrisk--2.38-0.87-diabetes}

\begin{enumerate}
\def\labelenumi{\alph{enumi}.}
\tightlist
\item
  \textbf{TRUE (Correct)}
\item
  FALSE
\end{enumerate}

\subsection{Correct rationales}\label{correct-rationales-4}

\begin{itemize}
\tightlist
\item
  The regression model:
  \[\log(risk) = \log(\mu_0) + \log(\theta)*Diabetes\] Then
  \[\log(\mu_0) = \log(0.093) = -2.38\] and
  \[\log(\theta) =\log(risk\,\,ratio) = \log(0.223/.093) = 0.87\]
\end{itemize}

\subsection{Incorrect rationales}\label{incorrect-rationales-4}

\section{The two risks can be connected via the equation:
log{[}risk/(1-risk){]} = -2.28 + 1.03 *
Diabetes}\label{the-two-risks-can-be-connected-via-the-equation-logrisk1-risk--2.28-1.03-diabetes}

\begin{enumerate}
\def\labelenumi{\alph{enumi}.}
\tightlist
\item
  \textbf{TRUE (Correct)}
\item
  FALSE
\end{enumerate}

\subsection{Correct rationales}\label{correct-rationales-5}

\begin{itemize}
\item
\end{itemize}

\begin{align*}
\log[R1/(1-R1)] &= \log [R0/(1-R0)] + \log {[R1/(1-R1)] /[R0/(1-R0)]} \cdot diabetes\\
\log [R1/(1-R1)] &= \log(9.3/90.7) + \log(2.8) \times diabetes 
\end{align*}

\subsection{Incorrect rationales}\label{incorrect-rationales-5}

\section{The two risks can be connected via the equation:
log{[}risk/(1-risk){]} = 1.03 - 2.28 *
Diabetes}\label{the-two-risks-can-be-connected-via-the-equation-logrisk1-risk-1.03---2.28-diabetes}

\begin{enumerate}
\def\labelenumi{\alph{enumi}.}
\tightlist
\item
  TRUE
\item
  \textbf{FALSE (Correct)}
\end{enumerate}

\subsection{Correct rationales}\label{correct-rationales-6}

\begin{itemize}
\tightlist
\item
  Coefficients reversed from previous question
\end{itemize}

\subsection{Incorrect rationales}\label{incorrect-rationales-6}

\section{4 year risk for intermittent claudication
I}\label{year-risk-for-intermittent-claudication-i}

Using the coefficients in his Table 12.2, Rothman (2012) calculated a 4
year risk for intermittent claudication of 6.7\% if the 70-year old man
had the example profile, which included normal blood pressure, AND
diabetes.

If the profile did NOT include diabetes, but was otherwise unchanged,
then (ignoring small rounding errors) the single equation that
connects/expresses/summarizes the risks for those without (reference
category) and with diabetes (Index category) can be written as (select
all that apply and justify your choices in the rationale)

\begin{enumerate}
\def\labelenumi{\alph{enumi}.}
\tightlist
\item
  \textbf{\%Risk{[}Diabetes{]} = 2.7 * (2.5 \^{} Diabetes) (Correct)}
\item
  \textbf{Odds{[}Diabetes{]} = (2.7/97.3) * ( ((6.7/93.3)/(2.7/97.3))
  \^{} Diabetes ) (Correct)}
\item
  \textbf{Odds{[}Diabetes{]} = (2.7/97.3) * (2.59 \^{} Diabetes)
  (Correct)}
\item
  \textbf{log{[} Odds{[}Diabetes{]} {]} = -3.58 + 0.95 * Diabetes
  (Correct)}
\item
  log{[} Risk{[}Diabetes{]} {]} = -3.58 + 0.95 * Diabetes
\end{enumerate}

\subsection{Correct rationales}\label{correct-rationales-7}

\begin{itemize}
\tightlist
\item
  A, B, C, and D are correct. If diabetes = 1 and no-diabetes = 0, the
  equation of A would work: 2.7 * (2.5 \^{} Diabetes) 2.7 * (2.5 \^{} 0)
  2.7 (1) = 2.7 = the risk among the unexposed, it's the intercept 2.7 *
  (2.5 \^{} Diabetes) 2.7 * (2.5 \^{} 1) 2.7 (1) = 6.7 = the risk among
  the exposed E is incorrect because when you calculate the log of
  theta, we get a value that is NOT 0.95.
\item
  A, B, C, and D are correct A is correct because when diabetes =0, \%
  risk = 2.7, and when diabetes =1, \% risk = 6.75\% which can be
  rounded down to 6.7\% E is wrong because even though Risk when no
  diabetes = 0.02717253 ln0.02717253 = 3.60 ln (risk ratio) =
  ln(6.7/2.7) = 0.909. This is very far off from 0.95 D is correct
  because: Odds when no diabetes = 0.0279 Ln odds no diabetes = -3.58
  Odds (diabetes =1) = 0.0718 , this aligns with the values from B
  (0.0718) and C(0.0719) for diabetes =1. Ln odds ratio = ln (0.0718/
  0.0279)=0.95
\end{itemize}

\subsection{Incorrect rationales}\label{incorrect-rationales-7}

\section{4 year risk for intermittent claudication
II}\label{year-risk-for-intermittent-claudication-ii}

Using the coefficients in his Table 12.2, Rothman (2012) calculated a 4
year risk for intermittent claudication of 6.7\% if the 70-year old man
had the example profile, which included normal blood pressure, AND
diabetes.

If we reversed the outcome and worked with an equation for STAYING FREE
of intermittent claudication, rather than developing it, the logistic
equations would

\begin{enumerate}
\def\labelenumi{\alph{enumi}.}
\tightlist
\item
  \textbf{have the same coefficients but all their signs would be
  reversed (Correct)}
\item
  have the same-sign intercept but the other coefficients would have
  their signs reversed
\item
  have the opposite-sign intercept but the same-sign other coefficients
\end{enumerate}

\subsection{Correct rationales}\label{correct-rationales-8}

\begin{itemize}
\tightlist
\item
  A is correct because we are predicting the opposite effect. B is
  incorrect because when all other predictors are 0, the risk of
  developing intermittent claudication cannot be the same as staying
  free of intermittent claudication. C is incorrect because the slopes
  for the predictors cannot be the same sign if we are predicting the
  opposite outcome.
\end{itemize}

\subsection{Incorrect rationales}\label{incorrect-rationales-8}

%\showmatmethods


\bibliography{pinp}
\bibliographystyle{jss}



\end{document}

