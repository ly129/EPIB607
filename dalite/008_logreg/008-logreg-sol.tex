\documentclass[letterpaper,9pt,twoside,printwatermark=false]{pinp}

%% Some pieces required from the pandoc template
\providecommand{\tightlist}{%
  \setlength{\itemsep}{0pt}\setlength{\parskip}{0pt}}

% Use the lineno option to display guide line numbers if required.
% Note that the use of elements such as single-column equations
% may affect the guide line number alignment.

\usepackage[T1]{fontenc}
\usepackage[utf8]{inputenc}

% The geometry package layout settings need to be set here...
\geometry{layoutsize={0.95588\paperwidth,0.98864\paperheight},%
          layouthoffset=0.02206\paperwidth,%
		  layoutvoffset=0.00568\paperheight}

\definecolor{pinpblue}{HTML}{185FAF}  % imagecolorpicker on blue for new R logo
\definecolor{pnasbluetext}{RGB}{101,0,0} %



\title{DALITE Q8 - Logistic Regression. Solutions.}

\author[a]{EPIB607 - Inferential Statistics}

  \affil[a]{Fall 2018, McGill University}

\setcounter{secnumdepth}{5}

% Please give the surname of the lead author for the running footer
\leadauthor{Bhatnagar and Hanley}

% Keywords are not mandatory, but authors are strongly encouraged to provide them. If provided, please include two to five keywords, separated by the pipe symbol, e.g:
 \keywords{  Two sample proportions |  Odds ratio |  Logit transformation |  Odds |  Log odds |  Risk calculator  }  

\begin{abstract}
This DALITE quiz will cover logistic regression.
\end{abstract}

\dates{This version was compiled on \today}
\doi{\url{https://sahirbhatnagar.com/EPIB607/}}

\pinpfootercontents{DALITE Q8 due November 18, 2018 by 11:59pm}

\begin{document}

% Optional adjustment to line up main text (after abstract) of first page with line numbers, when using both lineno and twocolumn options.
% You should only change this length when you've finalised the article contents.
\verticaladjustment{-2pt}

\maketitle
\thispagestyle{firststyle}
\ifthenelse{\boolean{shortarticle}}{\ifthenelse{\boolean{singlecolumn}}{\abscontentformatted}{\abscontent}}{}

% If your first paragraph (i.e. with the \dropcap) contains a list environment (quote, quotation, theorem, definition, enumerate, itemize...), the line after the list may have some extra indentation. If this is the case, add \parshape=0 to the end of the list environment.


\section*{FINRISK Calculator}\label{finrisk-calculator}
\addcontentsline{toc}{section}{FINRISK Calculator}

There are now several risk calculators derived from epidemiological
studies, such as the Framingham Heart Study. See, e.g.,
\url{https://www.uptodate.com/home/medical-calculators} and
\url{https://qxmd.com/calculate/}.

Many of the online ones for 10 year risks are based on the proportional
hazards model. One risk equation that is based on logistic regression
(see Rothman chapter 12, 2012 or Chapter 10, 2002) is that described in
the article Predicting Coronary Heart Disease and Stroke The FINRISK
Calculator by Erkki Vartiainen et al. GLOBAL HEART, VOL. 11, NO. 2, 2016
June 2016: 213-216 {[}article in myCourses{]}

By applying the equation in Table 2 of the article for the 10 year risk
of coronary heart disease for 55 year old females who don't smoke, but
whose cholesterol is 6, systolic blood pressure is 140, LDL is 0.5, are
NOT diabetic, and do not have a family history of MI, one obtains 9.3\%.

The same equation applied to females with the same profile, except that
they ARE diabetic, yields a 10 year risk of 22.3\%.

Thus, with respect to the No Diabetes (ref. category) vs.~Diabetes
(index category) contrast, say if the following statement is TRUE or
FALSE and explain why in your rationale:

\section{The risk difference is 13\%}\label{the-risk-difference-is-13}

\begin{enumerate}
\def\labelenumi{\alph{enumi}.}
\tightlist
\item
  TRUE
\item
  FALSE
\end{enumerate}

\subsection{Correct rationales}\label{correct-rationales}

\subsection{Incorrect rationales}\label{incorrect-rationales}

\section{The odds ratio is 2.4}\label{the-odds-ratio-is-2.4}

\begin{enumerate}
\def\labelenumi{\alph{enumi}.}
\tightlist
\item
  TRUE
\item
  FALSE
\end{enumerate}

\subsection{Correct rationales}\label{correct-rationales-1}

\subsection{Incorrect rationales}\label{incorrect-rationales-1}

\section{The risk ratio is 2.8}\label{the-risk-ratio-is-2.8}

\begin{enumerate}
\def\labelenumi{\alph{enumi}.}
\tightlist
\item
  TRUE
\item
  FALSE
\end{enumerate}

\subsection{Correct rationales}\label{correct-rationales-2}

\subsection{Incorrect rationales}\label{incorrect-rationales-2}

\section{The two risks can be connected via the equation: risk(\%) =
22.3 - 13 *
Diabetes}\label{the-two-risks-can-be-connected-via-the-equation-risk-22.3---13-diabetes}

\begin{enumerate}
\def\labelenumi{\alph{enumi}.}
\tightlist
\item
  TRUE
\item
  FALSE
\end{enumerate}

\subsection{Correct rationales}\label{correct-rationales-3}

\subsection{Incorrect rationales}\label{incorrect-rationales-3}

\section{The two risks can be connected via the equation: log{[}risk{]}
= -2.38 + 0.87 *
Diabetes}\label{the-two-risks-can-be-connected-via-the-equation-logrisk--2.38-0.87-diabetes}

\begin{enumerate}
\def\labelenumi{\alph{enumi}.}
\tightlist
\item
  TRUE
\item
  FALSE
\end{enumerate}

\subsection{Correct rationales}\label{correct-rationales-4}

\subsection{Incorrect rationales}\label{incorrect-rationales-4}

\section{The two risks can be connected via the equation:
log{[}risk/(1-risk){]} = -2.28 + 1.03 *
Diabetes}\label{the-two-risks-can-be-connected-via-the-equation-logrisk1-risk--2.28-1.03-diabetes}

\begin{enumerate}
\def\labelenumi{\alph{enumi}.}
\tightlist
\item
  TRUE
\item
  FALSE
\end{enumerate}

\subsection{Correct rationales}\label{correct-rationales-5}

\subsection{Incorrect rationales}\label{incorrect-rationales-5}

\section{The two risks can be connected via the equation:
log{[}risk/(1-risk){]} = 1.03 - 2.28 *
Diabetes}\label{the-two-risks-can-be-connected-via-the-equation-logrisk1-risk-1.03---2.28-diabetes}

\begin{enumerate}
\def\labelenumi{\alph{enumi}.}
\tightlist
\item
  TRUE
\item
  FALSE
\end{enumerate}

\subsection{Correct rationales}\label{correct-rationales-6}

\subsection{Incorrect rationales}\label{incorrect-rationales-6}

\section{4 year risk for intermittent
claudication}\label{year-risk-for-intermittent-claudication}

Using the coefficients in his Table 12.2, Rothman (2012) calculated a 4
year risk for intermittent claudication of 6.7\% if the 70-year old man
had the example profile, which included normal blood pressure, AND
diabetes.

If the profile did NOT include diabetes, but was otherwise unchanged,
then (ignoring small rounding errors) the single equation that
connects/expresses/summarizes the risks for those without (reference
category) and with diabetes (Index category) can be written as (select
all that apply and justify your choices in the rationale)

\begin{enumerate}
\def\labelenumi{\alph{enumi}.}
\tightlist
\item
  \%Risk{[}Diabetes{]} = 2.7 * (2.5 \^{} Diabetes)
\item
  Odds{[}Diabetes{]} = (2.7/97.3) * ( ((6.7/93.3)/(2.7/97.3)) \^{}
  Diabetes )
\item
  Odds{[}Diabetes{]} = (2.7/97.3) * (2.59 \^{} Diabetes)
\item
  log{[} Odds{[}Diabetes{]} {]} = -3.58 + 0.95 * Diabetes
\item
  log{[} Risk{[}Diabetes{]} {]} = -3.58 + 0.95 * Diabetes
\end{enumerate}

\subsection{Correct rationales}\label{correct-rationales-7}

\subsection{Incorrect rationales}\label{incorrect-rationales-7}

%\showmatmethods


\bibliography{pinp}
\bibliographystyle{jss}



\end{document}

