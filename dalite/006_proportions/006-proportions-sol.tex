\documentclass[letterpaper,9pt,twoside,printwatermark=false]{pinp}

%% Some pieces required from the pandoc template
\providecommand{\tightlist}{%
  \setlength{\itemsep}{0pt}\setlength{\parskip}{0pt}}

% Use the lineno option to display guide line numbers if required.
% Note that the use of elements such as single-column equations
% may affect the guide line number alignment.

\usepackage[T1]{fontenc}
\usepackage[utf8]{inputenc}

% The geometry package layout settings need to be set here...
\geometry{layoutsize={0.95588\paperwidth,0.98864\paperheight},%
          layouthoffset=0.02206\paperwidth,%
		  layoutvoffset=0.00568\paperheight}

\definecolor{pinpblue}{HTML}{185FAF}  % imagecolorpicker on blue for new R logo
\definecolor{pnasbluetext}{RGB}{101,0,0} %



\title{DALITE Q6 - Binomial Distribution and Inference for one Proportion.
Solutions.}

\author[a]{EPIB607 - Inferential Statistics}

  \affil[a]{Fall 2018, McGill University}

\setcounter{secnumdepth}{5}

% Please give the surname of the lead author for the running footer
\leadauthor{Bhatnagar and Hanley}

% Keywords are not mandatory, but authors are strongly encouraged to provide them. If provided, please include two to five keywords, separated by the pipe symbol, e.g:
 \keywords{  Binomial distribution |  One sample proportion |  Hypothesis testing  }  

\begin{abstract}
This DALITE quiz will cover the binomial distribution and inference for
a sample proportion. You need to understand the binomial distribution
before moving on to the chapter on one sample proportions. This is
analgous to learning the normal distribution before inference for a
single mean.
\end{abstract}

\dates{This version was compiled on \today}
\doi{\url{https://sahirbhatnagar.com/EPIB607/}}

\pinpfootercontents{DALITE Q5 due October 3, 2018 by 5pm}

\begin{document}

% Optional adjustment to line up main text (after abstract) of first page with line numbers, when using both lineno and twocolumn options.
% You should only change this length when you've finalised the article contents.
\verticaladjustment{-2pt}

\maketitle
\thispagestyle{firststyle}
\ifthenelse{\boolean{shortarticle}}{\ifthenelse{\boolean{singlecolumn}}{\abscontentformatted}{\abscontent}}{}

% If your first paragraph (i.e. with the \dropcap) contains a list environment (quote, quotation, theorem, definition, enumerate, itemize...), the line after the list may have some extra indentation. If this is the case, add \parshape=0 to the end of the list environment.


\section{Binomial Distribution 1}\label{binomial-distribution-1}

In which of the following would Y not have a Binomial distribution?
Provide your justification in the rationale.

\begin{enumerate}
\def\labelenumi{\alph{enumi}.}
\tightlist
\item
  \textbf{Y = Number, out of 60 occupants of 30 randomly chosen cars,
  wearing seatbelts. (Correct)}
\item
  Y = Number, out of 60 occupants of 60 randomly chosen cars, wearing
  seatbelts.
\item
  Y = Number, out of simple random sample of 100 individuals, that are
  left-handed.
\end{enumerate}

\subsection{Correct rationales}\label{correct-rationales}

\begin{itemize}
\tightlist
\item
  60 occupants from 30 cars would mean more than one occupant per car -
  whether each occupant is wearing a seatbelt or not isn't independent
  if more than one comes from the same car.
\item
  There is a chance that if someone in the car is not wearing a
  seatbelt, the other passenger in the car is not too. This means that
  it is not independent. \#idiotswhodon'twearseatbeltsdrivetogether
\item
  Not independent, idiots who don't wear seatbelts in cars drive
  together!!! VROOM VROOM
\item
  If there are more than one occupant in a car, they may influence each
  other to wear or not wear a seat belt.

  \begin{itemize}
  \tightlist
  \item
    Richard: ``hey Timmy, only losers wear seat belts, be cool like me
    and don't wear your seat belt''
  \item
    Timothy: ``oh snap, you're right Ricky, not wearing a seat belt is
    the fleekest''
  \item
    Richard and Timothy: ``we're going to live forever''
  \end{itemize}
\item
  The outcome does not fall into one of two categories as someone can be
  left-handed, right-handed, or ambidextrous.
\end{itemize}

\subsection{Incorrect rationales}\label{incorrect-rationales}

\begin{itemize}
\tightlist
\item
  Each trial does not have an equal probability of success as there are
  more right-handed people than left-handed people.
\item
  Because in this case the observations are not independent. If the
  first volunteer selected is left-handed, the second one is more likely
  to be right-handed because there are more right-handed individuals
  than left-handed in the remaining sample.
\end{itemize}

\section{Binomial Distribution 2}\label{binomial-distribution-2}

In which of the following would Y not have a Binomial distribution?
Provide your justification in the rationale.

\begin{enumerate}
\def\labelenumi{\alph{enumi}.}
\tightlist
\item
  \textbf{You want to know what percent of married people believe that
  mothers of young children should not be employed outside the home. You
  plan to interview 50 people, and for the sake of convenience you
  decide to interview both the husband and the wife in 25 married
  couples. The random variable Y is the number among the 50 persons
  interviewed who think mothers should not be employed. (Correct)}
\item
  You observe the sex of the next 50 children born at a local hospital;
  Y is the number of girls among them.
\item
  Y = number of occasions, out of a randomly selected sample of 100
  occasions during the year, in which you were indoors.
\end{enumerate}

\subsection{Correct rationales}\label{correct-rationales-1}

\begin{itemize}
\tightlist
\item
  Husband's opinion may depend on wife's opinion, and vice versa -
  observations aren't independent.
\item
  One of the conditions for the binomial distribution is that the trials
  must be independent. In A, they are selecting 25 married couples in
  order to gather data on 50 people, but in this scenario either the
  husband or the wife could influence the other, so the trials are not
  independent.
\end{itemize}

\subsection{Incorrect rationales}\label{incorrect-rationales-1}

\section{Binomial Distribution 3}\label{binomial-distribution-3}

The U.S. National Center for Health Statistics reports that
approximately 12\% of emergency department visits result in hospital
admissions. Consider 20 randomly selected emergency department visits
and assume that visits to emergency departments are independent. What is
the approximate probability that at most 2 of the 20 visits would result
in hospital admissions?

\begin{enumerate}
\def\labelenumi{\alph{enumi}.}
\tightlist
\item
  \textbf{0.5631 (Correct)}
\item
  0.2740
\item
  0.1344
\item
  0.2891
\item
  0.4369
\end{enumerate}

\subsection{Correct rationales}\label{correct-rationales-2}

\begin{itemize}
\tightlist
\item
  We have to callculate the probability at 0,1, and 2 and add them up:
  For 0: (1)(1)(0.88)\^{}20 For 1: 20(0.12)(0.88)\^{}19 For 2:
  190(0.12)\textsuperscript{2(0.88)}18 Adding these all up equals 0.5631
\item
  P(X\textless{} or=2) = P(X=2)+P(X=1)+P(X=0) = 0.5631
\item
  pbinom(2, 20, 0.12, lower.tail = TRUE)
\end{itemize}

\subsection{Incorrect rationales}\label{incorrect-rationales-2}

\begin{itemize}
\tightlist
\item
  1- pbinom(2, size = 20, prob = 0.12)
\end{itemize}

\section{Binomial DIstribution 4}\label{binomial-distribution-4}

The U.S. National Center for Health Statistics reports that
approximately 12\% of emergency department visits result in hospital
admissions. Consider 20 randomly selected emergency department visits
and assume that visits to emergency departments are independent. How
many hospital admissions do we expect, on average, in a random sample of
20 emergency department visits?

\begin{enumerate}
\def\labelenumi{\alph{enumi}.}
\tightlist
\item
  2
\item
  \textbf{2.4 (Correct)}
\item
  12
\item
  24
\item
  1.4533
\end{enumerate}

\subsection{Correct rationales}\label{correct-rationales-3}

\begin{itemize}
\tightlist
\item
  mean of a binomial distribution = np = 20*0.12 = 2.4
\item
  \(\mu\) = np = (20)(0.12) = 2.4 hospital admissions
\end{itemize}

\subsection{Incorrect rationales}\label{incorrect-rationales-3}

\begin{itemize}
\tightlist
\item
  0.12 x 20 = 2.4 But we can't in a random sample get .4, so we choose
  2.
\end{itemize}

\section{Binomial Distribution 5}\label{binomial-distribution-5}

The U.S. National Center for Health Statistics reports that
approximately 12\% of emergency department visits result in hospital
admissions. Consider 20 randomly selected emergency department visits
and assume that visits to emergency departments are independent. If Y is
the number of emergency department visits that result in hospital
admissions in random samples of 20 visits, what is approximately the
standard deviation of Y?

\begin{enumerate}
\def\labelenumi{\alph{enumi}.}
\tightlist
\item
  \textbf{1.45 (Correct)}
\item
  2.11
\item
  4.46
\end{enumerate}

\subsection{Correct rationales}\label{correct-rationales-4}

\begin{itemize}
\tightlist
\item
  The standard deviation of U can be solved using the equation sigma =
  sqrt(np(1-p)) = sqrt(20\emph{.12}(1-0.12)) = 1.45
\end{itemize}

\subsection{Incorrect rationales}\label{incorrect-rationales-4}

\section{Binomial Distribution 6}\label{binomial-distribution-6}

According to the 2015 U.S. census update, approximately 13\% of
Americans are black. Let Y be the number of blacks in a random sample of
1500 Americans. What is the probability that the sample will contain 200
or more blacks?

\begin{enumerate}
\def\labelenumi{\alph{enumi}.}
\tightlist
\item
  pbinom(q = 200, size = 1500, prob = 0.13)
\item
  1-dbinom(x = 200, size = 1500, prob = 0.13)
\item
  dbinom(x = 200, size = 1500, prob = 0.13)
\item
  \texttt{pnorm(200, mean = 195, sd = 13.025, lower.tail = FALSE)}
  \textbf{(Correct)}
\item
  \texttt{1 - pbinom(q = 200, size = 1500, prob = 0.13, lower.tail = FALSE)}
  \textbf{(Correct)}
\end{enumerate}

\subsection{Correct rationales}\label{correct-rationales-5}

\begin{itemize}
\item
  \begin{enumerate}
  \def\labelenumi{\alph{enumi})}
  \tightlist
  \item
    is not correct because it is looking at all values below 200 b) is
    not correct because it is looking at the probability of getting
    anything but 200 c) is not correct because it is looking at getting
    exactly 200 e) is not correct because it is looking at all values
    below \textbf{201} d) since sample size is large, we can use normal
    approximation for binomial distribution
  \end{enumerate}
\item
  In this case, we can use the normal approximation because n is large
  enough, n*p = 195 \textgreater{} 10, n(1-p) = 1305 \textgreater{} 10.
  Therefore, we can use pnorm function with q=200, lower.tail false. A
  doesn't specify a lower.tail=FALSE, so R gives lower.tail = TRUE by
  default. dbinom are wrong.
\end{itemize}

\subsection{Incorrect rationales}\label{incorrect-rationales-5}

\begin{itemize}
\item
  because you are calculating the sample that will contain 200 or more,
  lower.tail is false.
\item
  \begin{enumerate}
  \def\labelenumi{\Alph{enumi})}
  \setcounter{enumi}{4}
  \tightlist
  \item
    would be right if it didn't have the 1 - before the code.
  \end{enumerate}
\item
  Enough ppl not to do a t-test
\end{itemize}

\section{One Sample Proportion 1}\label{one-sample-proportion-1}

A CDC report on secondhand smoke at home gives the following 95\%
confidence interval for the percent of California households that are
free of secondhand smoke: (90.8, 92.2). The correct interpretation for
this confidence interval is that we can be 95\% confident that

\begin{enumerate}
\def\labelenumi{\alph{enumi}.}
\tightlist
\item
  the proportion of households free of secondhand smoke in another
  sample of California households would be between 0.908 and 0.922
\item
  the population mean number of households in California that are free
  of secondhand smoke is between 90.8 and 92.2
\item
  \textbf{the true proportion of all California households that are free
  of secondhand smoke is between 0.908 and 0.922 (Correct)}
\end{enumerate}

\subsection{Correct rationales}\label{correct-rationales-6}

\begin{itemize}
\tightlist
\item
  The answer is C because the confidence interval looks at where the
  true population proportion will fall 95\% of the time.
\item
  A is not right because the CI is only for that trial, not for all
  trials. B is not right because they are not talking about a mean value
  but a proportion.
\end{itemize}

\subsection{Incorrect rationales}\label{incorrect-rationales-6}

\section{One Sample Proportion 2}\label{one-sample-proportion-2}

A CDC report on secondhand smoke at home gives the following 95\%
confidence interval for the percent of California households that are
free of secondhand smoke: (90.8, 92.2). What is the margin of error for
this interval?

\begin{enumerate}
\def\labelenumi{\alph{enumi}.}
\tightlist
\item
  \textbf{0.007 (Correct)}
\item
  0.014
\item
  0.028
\end{enumerate}

\subsection{Correct rationales}\label{correct-rationales-7}

\begin{itemize}
\tightlist
\item
  The margin of error is half the width of the confidence interval.
\item
  The margin of error is half of the confidence interval. 92.2 - 90.8 /
  2 = 0.007
\end{itemize}

\subsection{Incorrect rationales}\label{incorrect-rationales-7}

\section{One Sample Proportion 3}\label{one-sample-proportion-3}

How many observations must be recorded to estimate a population with
unknown proportion p to within +/- 0.02 with 95\% confidence?

\begin{enumerate}
\def\labelenumi{\alph{enumi}.}
\tightlist
\item
  n=25
\item
  n=1225
\item
  \textbf{n=2401 (Correct)}
\item
  n=2350
\item
  n=1691
\end{enumerate}

\subsection{Correct rationales}\label{correct-rationales-8}

\begin{itemize}
\tightlist
\item
  We would have to guess p in this case because it is not given. It is
  best to assume p is 0.5 when it is unknown because the margin of error
  is largest when p is 0.5 so by doing so we are giving a more
  conservation estimation for how many people we will need. We then use
  the formula n=(z/m)\^{}2(p)(1-p) where p=0.5, z=1.96, m=0.02.
\end{itemize}

\subsection{Incorrect rationales}\label{incorrect-rationales-8}

\begin{itemize}
\tightlist
\item
  1691 = (1.645/0.02)\^{}2 * (0.5 * (1-0.5))
\item
  Using the formula: n = (z*/m)(p)(1-p) = (1.96/0.02)(0.5)(1-0.5) = 24.5
  = 25. Here, test is conservative and p is set to 0.5.
\end{itemize}

%\showmatmethods


\bibliography{pinp}
\bibliographystyle{jss}



\end{document}

