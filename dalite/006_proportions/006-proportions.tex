\documentclass[letterpaper,9pt,twocolumn,twoside,printwatermark=false]{pinp}

%% Some pieces required from the pandoc template
\providecommand{\tightlist}{%
  \setlength{\itemsep}{0pt}\setlength{\parskip}{0pt}}

% Use the lineno option to display guide line numbers if required.
% Note that the use of elements such as single-column equations
% may affect the guide line number alignment.

\usepackage[T1]{fontenc}
\usepackage[utf8]{inputenc}

% The geometry package layout settings need to be set here...
\geometry{layoutsize={0.95588\paperwidth,0.98864\paperheight},%
          layouthoffset=0.02206\paperwidth,%
		  layoutvoffset=0.00568\paperheight}

\definecolor{pinpblue}{HTML}{185FAF}  % imagecolorpicker on blue for new R logo
\definecolor{pnasbluetext}{RGB}{101,0,0} %



\title{DALITE Q6 - Binomial Distribution and Inference for Proportions. Due
October 17, 2018 by 5pm.}

\author[a]{EPIB607 - Inferential Statistics}

  \affil[a]{Fall 2018, McGill University}

\setcounter{secnumdepth}{5}

% Please give the surname of the lead author for the running footer
\leadauthor{Bhatnagar and Hanley}

% Keywords are not mandatory, but authors are strongly encouraged to provide them. If provided, please include two to five keywords, separated by the pipe symbol, e.g:
 \keywords{  Binomial distribution |  One sample proportion |  Hypothesis testing  }  

\begin{abstract}
This DALITE quiz will cover the binomial distribution and inference for
a sample proportion. You need to understand the binomial distribution
before moving on to the chapter on one sample proportions. This is
analgous to learning the normal distribution before inference for a
single mean.
\end{abstract}

\dates{This version was compiled on \today}
\doi{\url{https://sahirbhatnagar.com/EPIB607/}}

\pinpfootercontents{DALITE Q6 due October 17, 2018 by 5pm}

\begin{document}

% Optional adjustment to line up main text (after abstract) of first page with line numbers, when using both lineno and twocolumn options.
% You should only change this length when you've finalised the article contents.
\verticaladjustment{-2pt}

\maketitle
\thispagestyle{firststyle}
\ifthenelse{\boolean{shortarticle}}{\ifthenelse{\boolean{singlecolumn}}{\abscontentformatted}{\abscontent}}{}

% If your first paragraph (i.e. with the \dropcap) contains a list environment (quote, quotation, theorem, definition, enumerate, itemize...), the line after the list may have some extra indentation. If this is the case, add \parshape=0 to the end of the list environment.


\section*{Marking}\label{marking}
\addcontentsline{toc}{section}{Marking}

Completion of this DALITE exercise will be availble to us automatically
through the DALITE website. Therefore \textbf{you do not need to hand
anything in}. Marks will be based on the number of correct answers. For
each question you will receive 0.5 marks for getting the correct answer
on the first attempt and an additional 0.5 marks if you stick with the
right answer or switch to the correct answer after seeing someone else's
rationale. Recall that access to your assignments is managed through
tokens sent to your e-mail address. You will be sent a new link
everytime a new assignment has been posted.

\section{Binomial Distributions}\label{binomial-distributions}

\subsection{Learning Objectives}\label{learning-objectives}

\begin{enumerate}
\def\labelenumi{\arabic{enumi}.}
\tightlist
\item
  Be able to identify a binomial setting and define a binomial random
  variable.
\item
  Know how to find probabilities associated with a binomial random
  variable.
\item
  Know how to determine the mean and standard deviation of a binomial
  random variable.
\end{enumerate}

\subsection{Videos}\label{videos}

\begin{enumerate}
\def\labelenumi{\arabic{enumi}.}
\tightlist
\item
  \href{https://www.learner.org/courses/againstallodds/unitpages/unit21.html}{Against
  All Odds Unit 21}
\end{enumerate}

\subsection{Required Readings}\label{required-readings}

\begin{enumerate}
\item \href{https://www.learner.org/courses/againstallodds/pdfs/AgainstAllOdds_StudentGuide_Unit21.pdf#page=1}{Against All Odds Unit 21, pages 1-10}
\item \href{https://www.dropbox.com/s/g1ui3xg8nup1p6b/proportion-model-inference-plan-2018.pdf?dl=0}{JH notes Section 1 on binomial distributions}
\item \href{https://www.dropbox.com/s/gse9zpx4v5f3lhb/Ch12DescreteDistributions.pdf?dl=0}{B\&M chapter 12, pages 289-307}
\end{enumerate}

\vspace*{0.25cm}

\section{Inference for Proportions}\label{inference-for-proportions}

\subsection{Learning Objectives}\label{learning-objectives-1}

\begin{enumerate}
\def\labelenumi{\arabic{enumi}.}
\tightlist
\item
  Identify inference problems that concern a population proportion.
\item
  Know how to conduct a significance test of a population proportion.
\item
  Be able to calculate a confidence interval for a population
  proportion.
\item
  Understand that the \(z\)-inference procedures for proportions are
  based on approximations to the normal distribution and that accuracy
  depends on having moderately large sample sizes.
\end{enumerate}

\subsection{Videos}\label{videos-1}

\begin{enumerate}
\def\labelenumi{\arabic{enumi}.}
\tightlist
\item
  \href{https://www.learner.org/courses/againstallodds/unitpages/unit28.html}{Against
  All Odds Unit 28}
\end{enumerate}

\subsection{Required Readings}\label{required-readings-1}

\begin{enumerate}
\item \href{https://www.learner.org/courses/againstallodds/pdfs/AgainstAllOdds_StudentGuide_Unit28.pdf#page=1}{Against All Odds Unit 28, pages 1-11}
\item \href{https://www.dropbox.com/s/g1ui3xg8nup1p6b/proportion-model-inference-plan-2018.pdf?dl=0}{JH notes Section 2 on Inference for a proportion}
\item \href{https://www.dropbox.com/s/2h86m5qb70ntmcg/Ch19InferenceAboutPopulationProportion.pdf?dl=0}{B\&M chapter 19, pages 463-479}
\end{enumerate}

%\showmatmethods


\bibliography{pinp}
\bibliographystyle{jss}



\end{document}

