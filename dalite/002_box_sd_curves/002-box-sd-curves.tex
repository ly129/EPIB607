\documentclass[letterpaper,9pt,twocolumn,twoside,printwatermark=false]{pinp}

%% Some pieces required from the pandoc template
\providecommand{\tightlist}{%
  \setlength{\itemsep}{0pt}\setlength{\parskip}{0pt}}

% Use the lineno option to display guide line numbers if required.
% Note that the use of elements such as single-column equations
% may affect the guide line number alignment.

\usepackage[T1]{fontenc}
\usepackage[utf8]{inputenc}

% The geometry package layout settings need to be set here...
\geometry{layoutsize={0.95588\paperwidth,0.98864\paperheight},%
          layouthoffset=0.02206\paperwidth,%
		  layoutvoffset=0.00568\paperheight}

\definecolor{pinpblue}{HTML}{185FAF}  % imagecolorpicker on blue for new R logo
\definecolor{pnasbluetext}{RGB}{101,0,0} %



\title{DALITE Q2 - Boxplots, Standard Deviation and Normal Curves. Due
September 19, 2018.}

\author[a]{EPIB607 - Inferential Statistics}

  \affil[a]{Fall 2018, McGill University}

\setcounter{secnumdepth}{5}

% Please give the surname of the lead author for the running footer
\leadauthor{Bhatnagar and Hanley}

% Keywords are not mandatory, but authors are strongly encouraged to provide them. If provided, please include two to five keywords, separated by the pipe symbol, e.g:
 \keywords{  Boxplots |  Standard deviation |  Normal curves  }  

\begin{abstract}
This DALITE quiz will cover more descriptives such as boxplots, standard
deviation, and introduce you to normal density curves.
\end{abstract}

\dates{This version was compiled on \today}
\doi{\url{https://sahirbhatnagar.com/EPIB607/}}

\pinpfootercontents{DALITE Q2 due Sepetember 19, 2018 by 5pm}

\begin{document}

% Optional adjustment to line up main text (after abstract) of first page with line numbers, when using both lineno and twocolumn options.
% You should only change this length when you've finalised the article contents.
\verticaladjustment{-2pt}

\maketitle
\thispagestyle{firststyle}
\ifthenelse{\boolean{shortarticle}}{\ifthenelse{\boolean{singlecolumn}}{\abscontentformatted}{\abscontent}}{}

% If your first paragraph (i.e. with the \dropcap) contains a list environment (quote, quotation, theorem, definition, enumerate, itemize...), the line after the list may have some extra indentation. If this is the case, add \parshape=0 to the end of the list environment.


\section*{Marking}\label{marking}
\addcontentsline{toc}{section}{Marking}

Completion of this DALITE exercise will be availble to us automatically
through the DALITE website. Therefore \textbf{you do not need to hand
anything in}. Marks will be based on the number of correct answers. For
each question you will receive 0.5 marks for getting the correct answer
on the first attempt and an additional 0.5 marks if you stick with the
right answer or switch to the correct answer after seeing someone else's
rationale. Recall that access to your assignments is managed through
tokens sent to your e-mail address. You will be sent a new link
everytime a new assignment has been posted.

\section{Boxplots}\label{boxplots}

\subsection{Learning Objectives}\label{learning-objectives}

\begin{enumerate}
\def\labelenumi{\arabic{enumi}.}
\tightlist
\item
  Recognize that a basic numerical description of a distribution
  requires both a measure of center and a measure of spread.
\item
  Use the quartiles and the extremes to provide information about the
  unequal spread in the two sides of a skewed distribution.
\item
  Be able to calculate the quartiles and give the five-number summary of
  a data set of using a computer.
\item
  Understand that boxplots provide less detail than stemplots or
  histograms but are especially useful for comparing several
  distributions.
\end{enumerate}

\subsection{Videos}\label{videos}

\begin{enumerate}
\def\labelenumi{\arabic{enumi}.}
\tightlist
\item
  \href{https://www.learner.org/courses/againstallodds/unitpages/unit05.html}{Against
  All Odds Unit 5}
\end{enumerate}

\subsection{Required Readings}\label{required-readings}

\begin{enumerate}
\def\labelenumi{\arabic{enumi}.}
\tightlist
\item
  \href{https://www.learner.org/courses/againstallodds/pdfs/AgainstAllOdds_StudentGuide_Unit03.pdf\#page=1}{Against
  All Odds Unit 5, pages 1-5}
\end{enumerate}

\vspace*{0.25cm}

\section{Standard Deviation}\label{standard-deviation}

\subsection{Learning Objectives}\label{learning-objectives-1}

\begin{enumerate}
\def\labelenumi{\arabic{enumi}.}
\tightlist
\item
  Know that the sample standard deviation, \(s\), is the measure of
  spread most commonly used when the mean, \(\bar{x}\), is used as the
  measure of center.
\item
  Know the formula for the standard deviation \(s\)
\item
  Know the basic properties of the standard deviation:

  \begin{enumerate}
  \def\labelenumii{\alph{enumii})}
  \tightlist
  \item
    \(s \geq 0\), and only when all data values are identical can
    \(s = 0\)
  \item
    \(s\) increases as the spread about \(x\) increases.
  \item
    \(s\), like \(\bar{x}\), is strongly influenced by outliers.
  \end{enumerate}
\item
  Know that the standard deviation is most useful for symmetric
  distributions and, in particular, for normal distributions.
\item
  Know that adding the same constant \(a\) to all the observations
  increases the value of \(\bar{x}\) by \(a\). However, adding the same
  constant \(a\) to all the observations does not change the value of
  \(s\). That's because adding a constant \(a\) to all data values
  shifts the location of the data but does not affect its spread.
\item
  Know that multiplying all data values by a constant amount \(k\)
  changes \(\bar{x}\) and \(s\) by a factor of \(k\).
\end{enumerate}

\subsection{Videos}\label{videos-1}

\begin{enumerate}
\def\labelenumi{\arabic{enumi}.}
\tightlist
\item
  \href{https://www.learner.org/courses/againstallodds/unitpages/unit06.html}{Against
  All Odds Unit 6}
\end{enumerate}

\subsection{Required Readings}\label{required-readings-1}

\begin{enumerate}
\item \href{https://www.learner.org/courses/againstallodds/pdfs/AgainstAllOdds_StudentGuide_Unit06.pdf#page=1}{Against All Odds Unit 6, pages 1-8}
\end{enumerate}

\section{Normal Curves}\label{normal-curves}

\subsection{Learning Objectives}\label{learning-objectives-2}

\begin{enumerate}
\def\labelenumi{\arabic{enumi}.}
\tightlist
\item
  Understand that the overall shape of a distribution of a large number
  of observations can be summarized by a smooth curve called a density
  curve.
\item
  Know that an area under a density curve over an interval represents
  the proportion of data that falls in that interval.
\item
  Recognize the characteristic bell-shapes of normal curves. Locate the
  mean and standard deviation on a normal density curve by eye.
\item
  Understand how changing the mean and standard deviation affects a
  normal density curve.

  \begin{itemize}
  \tightlist
  \item
    Know that changing the mean of a normal density curve shifts the
    curve along the horizontal axis without changing its shape.
  \item
    Know that increasing the standard deviation produces a flatter and
    wider bell-shaped curve and that decreasing the standard deviation
    produces a taller and narrower curve.
  \end{itemize}
\end{enumerate}

\subsection{Videos}\label{videos-2}

\begin{enumerate}
\def\labelenumi{\arabic{enumi}.}
\tightlist
\item
  \href{https://www.learner.org/courses/againstallodds/unitpages/unit07.html}{Against
  All Odds Unit 7}
\end{enumerate}

\subsection{Required Readings}\label{required-readings-2}

\begin{enumerate}
\item \href{https://www.learner.org/courses/againstallodds/pdfs/AgainstAllOdds_StudentGuide_Unit07.pdf#page=1}{Against All Odds Unit 7, pages 1-9}
\end{enumerate}

%\showmatmethods


\bibliography{pinp}
\bibliographystyle{jss}



\end{document}

