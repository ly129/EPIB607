\documentclass[letterpaper,9pt,twoside,printwatermark=false]{pinp}

%% Some pieces required from the pandoc template
\providecommand{\tightlist}{%
  \setlength{\itemsep}{0pt}\setlength{\parskip}{0pt}}

% Use the lineno option to display guide line numbers if required.
% Note that the use of elements such as single-column equations
% may affect the guide line number alignment.

\usepackage[T1]{fontenc}
\usepackage[utf8]{inputenc}

% The geometry package layout settings need to be set here...
\geometry{layoutsize={0.95588\paperwidth,0.98864\paperheight},%
          layouthoffset=0.02206\paperwidth,%
		  layoutvoffset=0.00568\paperheight}

\definecolor{pinpblue}{HTML}{185FAF}  % imagecolorpicker on blue for new R logo
\definecolor{pnasbluetext}{RGB}{101,0,0} %



\title{DALITE Q5 - Bootstrap, Tests of Significance and Small Sample Inference
for One Mean. Solutions.}

\author[a]{EPIB607 - Inferential Statistics}

  \affil[a]{Fall 2018, McGill University}

\setcounter{secnumdepth}{5}

% Please give the surname of the lead author for the running footer
\leadauthor{Bhatnagar and Hanley}

% Keywords are not mandatory, but authors are strongly encouraged to provide them. If provided, please include two to five keywords, separated by the pipe symbol, e.g:
 \keywords{  Hypothesis testing |  Bootstrap |  t distribution |  One sample mean |  Normal calculations |  Confidence intervals |  Central Limit Theorem (CLT)  }  

\begin{abstract}
This DALITE quiz will cover the bootstrap, an introduction to
significance testing, and inference for a single mean using the t
distribution.
\end{abstract}

\dates{This version was compiled on \today}
\doi{\url{https://sahirbhatnagar.com/EPIB607/}}

\pinpfootercontents{DALITE Q5 due October 3, 2018 by 5pm}

\begin{document}

% Optional adjustment to line up main text (after abstract) of first page with line numbers, when using both lineno and twocolumn options.
% You should only change this length when you've finalised the article contents.
\verticaladjustment{-2pt}

\maketitle
\thispagestyle{firststyle}
\ifthenelse{\boolean{shortarticle}}{\ifthenelse{\boolean{singlecolumn}}{\abscontentformatted}{\abscontent}}{}

% If your first paragraph (i.e. with the \dropcap) contains a list environment (quote, quotation, theorem, definition, enumerate, itemize...), the line after the list may have some extra indentation. If this is the case, add \parshape=0 to the end of the list environment.


\section{Hypothesis tests 1}\label{hypothesis-tests-1}

The average human gestation time is 266 days from conception. A
researcher suspects that proper nutrition plays an important role and
that poor women with inadequate food intake would have shorter gestation
times even when given vitamin supplements. A random sample of 20 poor
women given vitamin supplements throughout the pregnancy has a mean
gestation time from conception of \(\bar{y}\)=256 days. The null
hypothesis for the researcher's test is

\begin{enumerate}
\def\labelenumi{\alph{enumi}.}
\tightlist
\item
  \textbf{\(\textrm{H}_0\): \(\mu\) = 266 (Correct)}
\item
  \(\textrm{H}_0\): \(\mu\) = 256
\item
  \(\textrm{H}_0\): \(\mu\) \textless{} 266
\item
  \(\textrm{H}_0\): \(\bar{y}\) = 266
\end{enumerate}

\subsection{Correct rationales}\label{correct-rationales}

\begin{itemize}
\tightlist
\item
  The null hypothesis is usually a statement of ``no effect'' or ``no
  difference.''
\item
  The null hypothesis is the statement of no effect, so if there was no
  effect, the average gestation time would be the population mean.
\item
  The null hypothesis of this study is that the average gestation period
  of women with proper nutrition and poor women with inadequate food
  intake is the same that is 266 days. Hence, the population mean (all
  poor women) have the gestation period of 266 days.
\end{itemize}

\subsection{Incorrect rationales}\label{incorrect-rationales}

\begin{itemize}
\tightlist
\item
  The null hypothesis is ``the accepted fact'', which in this case is
  that poorer nutrion will yield results lower than the mean
  (\textless{}266) or H null: u\textless{}266
\item
  The researcher is testing that the hypothesis that poor women with
  inadequate food intake will have lower gestation time
\end{itemize}

\section{Hypothesis tests 2}\label{hypothesis-tests-2}

The average human gestation time is 266 days from conception. A
researcher suspects that proper nutrition plays an important role and
that poor women with inadequate food intake would have shorter gestation
times even when given vitamin supplements. A random sample of 20 poor
women given vitamin supplements throughout the pregnancy has a mean
gestation time from conception of \(\bar{y}\)=256 days. The researcher's
alternative hypothesis for the test is

\begin{enumerate}
\def\labelenumi{\alph{enumi}.}
\tightlist
\item
  \(\textrm{H}_a\): \(\mu \neq 256\)
\item
  \textbf{\(\textrm{H}_a\): \(\mu < 266\) (Correct)}
\item
  \(\textrm{H}_a\): \(\mu < 256\)
\end{enumerate}

\subsection{Correct rationales}\label{correct-rationales-1}

\begin{itemize}
\tightlist
\item
  The alternative hypothesis is the point of view of the researchers. In
  this scenario, the researchers are suspecting that there will be a
  shorter gestation time for poor women in comparison to an average
  woman. The alternative hypothesis should be one-sided because the
  investigators only want to know if gestation time will be shorter and
  not if it is overall different (shorter or greater) from the average
  gestation time.
\item
  Researcher has specified a specific direction - that the sample would
  generate a mean gestation time less than 266. Therefore the
  alternative hypothesis is that \(\mu\) is less than 266.
\end{itemize}

\subsection{Incorrect rationales}\label{incorrect-rationales-1}

\section{Hypothesis test 3}\label{hypothesis-test-3}

The average human gestation time is 266 days from conception. A
researcher suspects that proper nutrition plays an important role and
that poor women with inadequate food intake would have shorter gestation
times even when given vitamin supplements. A random sample of 20 poor
women given vitamin supplements throughout the pregnancy has a mean
gestation time from conception of \(\bar{y}\)=256 days. Human gestation
times are approximately Normal with standard deviation \(\sigma\) = 16
days. The p-value for the researcher's test is (provide your calculation
in the rationale)

\begin{enumerate}
\def\labelenumi{\alph{enumi}.}
\tightlist
\item
  more than 0.1
\item
  \textbf{less than 0.01 (Correct)}
\item
  less than 0.001
\item
  less than 0.05
\item
  less than 0.025
\end{enumerate}

\subsection{Correct rationales}\label{correct-rationales-2}

\begin{itemize}
\tightlist
\item
  \(Z= (y - \mu)/ (\sigma / sqrt (n))\) = -2.795 \(\to\)
  \texttt{pnorm(-2.795,\ mean\ =\ 0\ ,\ sd=1\ )\ =\ 0.00026}
\item
  \texttt{mosaic::xpnorm(q=\ (256\ -\ 266)/(16/sqrt(20)))}
\item
  \texttt{stats::pnorm(q\ =\ 256,\ mean\ =\ 266,\ sd\ =\ 16/(sqrt(20)),\ lower.tail\ =\ T)}
\end{itemize}

\subsection{Incorrect rationales}\label{incorrect-rationales-2}

\begin{itemize}
\tightlist
\item
  \(t* = (256-266) / (16 / \sqrt(20))\) = -2.795. degrees of freedom =
  n-1 = 19. p-value is between 0.01 and 0.005. This is a one tailed
  t-test so divide by 2 and p-value is between 0.005 and 0.0025.
  Therefore, less than 0.01.
\item
  \((256 - 266)/16 = -0.625 \to\) \texttt{pnorm(-0.625)} corresponds to
  a p value of 0.27
\item
  \(t = (\bar{y}-\mu)/(\sigma/\sqrt{n})\) then use \(t\) distribution
  with df=19
\end{itemize}

\section{Hypothesis tests 4}\label{hypothesis-tests-4}

Average human gestation time is 266 days, when counted from conception.
A hospital gives a 90\% confidence interval for the mean gestation time
from conception among its patients. That interval is 264 \(\pm\) 5 days.
Is the mean gestation time in that hospital significantly different from
266 days?

\begin{enumerate}
\def\labelenumi{\alph{enumi}.}
\tightlist
\item
  \textbf{It is not significantly different at the 10\% level and
  therefore is also not significantly different at the 5\% level.
  (Correct)}
\item
  It is not significantly different at the 10\% level but might be
  significantly different at the 5\% level. c. It is significantly
  different at the 10\% level
\end{enumerate}

\subsection{Correct rationales}\label{correct-rationales-3}

\begin{itemize}
\tightlist
\item
  The average human gestation time falls within the interval of the
  hospital which is 259-269 days, therefore the mean gestation time is
  not significantly different at the 10\% level and since the 5\% level
  is even more specific, then it is not significantly different at the
  5\% level either.
\item
  266 is included in the 259-269 range, therefore the mean gestation
  time in the hospital is not significantly different from 266 days at
  the 10\% level. Looking at the significance at the 5\% level would
  make the confidence interval bigger, meaning there's not significant
  different there either
\end{itemize}

\subsection{Incorrect rationales}\label{incorrect-rationales-3}

\begin{itemize}
\tightlist
\item
  Because 266 falls at the upper tail of the 90\% confidence interval,
  therefore it is significant different at 10\% significance level
\item
  It is significantly different at the 10\% level, since the CI does not
  contain the value zero.
\item
  The 90\% CI includes 266, but the 95\% CI may not include 266, in
  which case a value like 266 would only be observed less than or equal
  to 5\% of the time.
\item
  To be significantly different, p-value should be less than 5\%.
\item
  As 266 falls within 264+/- 5 days, it is not significant at the 10\%
  level, but it may not fall under this range because when the level of
  confidence is increased, the confidence interval becomes smaller.
\end{itemize}

\section{One sample mean}\label{one-sample-mean}

We prefer the \(t\) procedures to the \(z\) procedures for inference
about a population mean because

\begin{enumerate}
\def\labelenumi{\alph{enumi}.}
\tightlist
\item
  \(z\) can be used only for large samples
\item
  \textbf{\(z\) requires that you know the population standard deviation
  \(\sigma\) (Correct)}
\item
  \(z\) requires that you can regard your data as an SRS from the
  population
\end{enumerate}

\subsection{Correct rationales}\label{correct-rationales-4}

\begin{itemize}
\tightlist
\item
  The reason we can use z on large samples is because when we have large
  samples, we can use the sample sd as an estimate for the population
  sd. The core reason for using t is because we don't know the
  population sd and we can't estimate the sd with the small sample size.
\item
  because you don't know population sd, only sample sd for t procedures
\item
  Z test needs to have a known population standard deviation sigma and
  either a normal distribution or the sample size n is large. If the
  population standard deviation is unknown and the sample size is small
  then t test statistic can be used.
\end{itemize}

\subsection{Incorrect rationales}\label{incorrect-rationales-4}

\begin{itemize}
\tightlist
\item
  df are smaller for t-test, and you do not need a SD value
\end{itemize}

\section{One sample mean 2}\label{one-sample-mean-2}

Because \(t\) procedures are robust, the most important condition for
their safe use is that

\begin{enumerate}
\def\labelenumi{\alph{enumi}.}
\tightlist
\item
  the population standard deviation \(\sigma\) is known
\item
  the population distribution is exactly Normal
\item
  \textbf{the data can be regarded as an SRS from the population
  (Correct)}
\item
  the CLT hasn't kicked in yet
\item
  the sample size is small
\end{enumerate}

\subsection{Correct rationales}\label{correct-rationales-5}

\begin{itemize}
\tightlist
\item
  The point of the t test is that sigma is not known (rules out A) The
  CLT says that the population distribution doesn't matter (rules out B)
  We want the CLT to kick in (rules out D) The larger the sample size
  the better(rules out E).
\end{itemize}

\subsection{Incorrect rationales}\label{incorrect-rationales-5}

\begin{itemize}
\tightlist
\item
  We should only use t procedures if the CLT is not applicable.
\item
  In order to use the T procedure comfortably, the sample size must be
  large and the population distribution must be normal.
\end{itemize}

\section{Bootstrap}\label{bootstrap}

Which of the following statements \emph{best} describes the utility of
the bootstrap

\begin{enumerate}
\def\labelenumi{\alph{enumi}.}
\tightlist
\item
  The bootstrap frees us from the requirement of using simple formulas
  to derive confidence intervals
\item
  The bootstrap allows us to simulate a sampling distribution
\item
  \textbf{The bootstrap frees us from the assumption of a Gaussian
  sampling distribution for the mean (as per the CLT) (Correct)}
\item
  The bootstrap tells us if the sampling distribution is asymmetric
\end{enumerate}

\subsection{Correct rationales}\label{correct-rationales-6}

\begin{itemize}
\tightlist
\item
  Computer intensive methods can solve most problems without assuming
  that the data have a Gaussian distribution.
\item
  The bootstrap frees us from the assumption that the data conform to a
  bell-shaped normal distribution.
\end{itemize}

\subsection{Incorrect rationales}\label{incorrect-rationales-6}

\begin{itemize}
\tightlist
\item
  due to the nature of bootstrap, you can resample things MANY times,
  which means that CLT will always kick in giving a gaussian
  distribution instead of assuming one exists
\item
  Bootstrap is useful in because it does not require the CLT and
  \textbf{the population} does not have to be normally distributed
\item
  The bootstrap doesn't rely on CLT, and gives us freedom from having to
  assume the normal distribution of \textbf{the population}.
\item
  The bootstrap allows us to derive estimates when often-used theories
  do not apply. This is especially useful when n is small (e.g.~CLT
  requires n \textgreater{} 30 to assume normality).
\end{itemize}

%\showmatmethods


\bibliography{pinp}
\bibliographystyle{jss}



\end{document}

