\documentclass[letterpaper,9pt,twocolumn,twoside,printwatermark=false]{pinp}

%% Some pieces required from the pandoc template
\providecommand{\tightlist}{%
  \setlength{\itemsep}{0pt}\setlength{\parskip}{0pt}}

% Use the lineno option to display guide line numbers if required.
% Note that the use of elements such as single-column equations
% may affect the guide line number alignment.

\usepackage[T1]{fontenc}
\usepackage[utf8]{inputenc}

% The geometry package layout settings need to be set here...
\geometry{layoutsize={0.95588\paperwidth,0.98864\paperheight},%
          layouthoffset=0.02206\paperwidth,%
		  layoutvoffset=0.00568\paperheight}

\definecolor{pinpblue}{HTML}{185FAF}  % imagecolorpicker on blue for new R logo
\definecolor{pnasbluetext}{RGB}{101,0,0} %



\title{DALITE Q5 - Bootstrap, Tests of Significance and Small Sample Inference
for One Mean. Due October 3, 2018 by 5pm.}

\author[a]{EPIB607 - Inferential Statistics}

  \affil[a]{Fall 2018, McGill University}

\setcounter{secnumdepth}{5}

% Please give the surname of the lead author for the running footer
\leadauthor{Bhatnagar and Hanley}

% Keywords are not mandatory, but authors are strongly encouraged to provide them. If provided, please include two to five keywords, separated by the pipe symbol, e.g:
 \keywords{  Hypothesis testing |  Bootstrap |  t distribution |  One sample mean |  Normal calculations |  Confidence intervals |  Central Limit Theorem (CLT)  }  

\begin{abstract}
This DALITE quiz will cover the bootstrap, an introduction to
significance testing, and inference for a single mean using the t
distribution.
\end{abstract}

\dates{This version was compiled on \today}
\doi{\url{https://sahirbhatnagar.com/EPIB607/}}

\pinpfootercontents{DALITE Q5 due October 3, 2018 by 5pm}

\begin{document}

% Optional adjustment to line up main text (after abstract) of first page with line numbers, when using both lineno and twocolumn options.
% You should only change this length when you've finalised the article contents.
\verticaladjustment{-2pt}

\maketitle
\thispagestyle{firststyle}
\ifthenelse{\boolean{shortarticle}}{\ifthenelse{\boolean{singlecolumn}}{\abscontentformatted}{\abscontent}}{}

% If your first paragraph (i.e. with the \dropcap) contains a list environment (quote, quotation, theorem, definition, enumerate, itemize...), the line after the list may have some extra indentation. If this is the case, add \parshape=0 to the end of the list environment.


\section*{Marking}\label{marking}
\addcontentsline{toc}{section}{Marking}

Completion of this DALITE exercise will be availble to us automatically
through the DALITE website. Therefore \textbf{you do not need to hand
anything in}. Marks will be based on the number of correct answers. For
each question you will receive 0.5 marks for getting the correct answer
on the first attempt and an additional 0.5 marks if you stick with the
right answer or switch to the correct answer after seeing someone else's
rationale. Recall that access to your assignments is managed through
tokens sent to your e-mail address. You will be sent a new link
everytime a new assignment has been posted.

\section{Bootstrap}\label{bootstrap}

\subsection{Learning Objectives}\label{learning-objectives}

\begin{enumerate}
\def\labelenumi{\arabic{enumi}.}
\tightlist
\item
  Understand that the bootstrap can be used to simulate a sampling
  distribution
\item
  Confidence intervals can subsequently be calculated directly from the
  bootstrap distribution
\item
  Bootstrap confidence intervals do not rely on the Central Limit
  Theorem
\end{enumerate}

\subsection{Required Readings}\label{required-readings}

\begin{enumerate}
\item \href{https://www.dropbox.com/s/cxiq70zxxtyxlb5/EfronDiaconisBootstrap.pdf?dl=0}{Computer-Intensive Methods in Statistics by Persi Diaconis and Bradley Efron, Scientific American 1983}
\end{enumerate}

\vspace*{0.25cm}

\section{Tests of Significance}\label{tests-of-significance}

\subsection{Learning Objectives}\label{learning-objectives-1}

\begin{enumerate}
\def\labelenumi{\arabic{enumi}.}
\tightlist
\item
  Understand that a significance test answers the question ``Is this
  sample outcome good evidence that an effect is present in the
  population, or could it easily occur just by chance?''
\item
  Be able to formulate the null hypothesis and alternative hypothesis
  for tests about the mean of a population. Understand that the
  alternative hypothesis is the researcher's point of view.
\item
  Understand the concept of a p-value. Know that smaller p-values
  indicate stronger evidence against the null hypothesis.
\item
  Be able to calculate p-values as areas under a normal curve in the
  setting of tests about the mean of a normal population with known
  standard deviation.
\item
  Be able to test a population mean with a z-test.
\end{enumerate}

\subsection{Videos}\label{videos}

\begin{enumerate}
\def\labelenumi{\arabic{enumi}.}
\tightlist
\item
  \href{https://www.learner.org/courses/againstallodds/unitpages/unit25.html}{Against
  All Odds Unit 25}
\end{enumerate}

\subsection{Required Readings}\label{required-readings-1}

\begin{enumerate}
\item \href{https://www.learner.org/courses/againstallodds/pdfs/AgainstAllOdds_StudentGuide_Unit25.pdf#page=1}{Against All Odds Unit 25, pages 1-12}
\item \href{http://www.medicine.mcgill.ca/epidemiology/hanley/BionanoWorkshop/P-Values.pdf}{JH notes on p-values}
\item \href{https://www.dropbox.com/s/luzhlatvx9hvwyn/Ch14IntroToInference.pdf?dl=0}{B\&M chapter 14, pages 345-359}
\end{enumerate}

\section{Small Sample Inference for One
Mean}\label{small-sample-inference-for-one-mean}

\subsection{Learning Objectives}\label{learning-objectives-2}

\begin{enumerate}
\def\labelenumi{\arabic{enumi}.}
\tightlist
\item
  Understand when to use t-procedures for a single sample and how they
  differ from the z-procedures covered in Units 24 and 25.
\item
  Understand what a t-distribution is and how it differs from a normal
  distribution.
\item
  Know how to check whether the underlying assumptions for a t-test or
  t-confidence interval are reasonably satisfied.
\item
  Be able to calculate a t-confidence interval for a population mean.
\item
  Be able to test a population mean with a t-test. Be able to calculate
  the t-test statistic and to determine the p-value as an area under a
  t-density curve.
\item
  Be able to adapt one-sample t-procedures to analyze matched pairs
  data.
\end{enumerate}

\subsection{Videos}\label{videos-1}

\begin{enumerate}
\def\labelenumi{\arabic{enumi}.}
\tightlist
\item
  \href{https://www.learner.org/courses/againstallodds/unitpages/unit26.html}{Against
  All Odds Unit 26}
\end{enumerate}

\subsection{Required Readings}\label{required-readings-2}

\begin{enumerate}
\item \href{https://www.learner.org/courses/againstallodds/pdfs/AgainstAllOdds_StudentGuide_Unit26.pdf#page=1}{Against All Odds Unit 26, pages 1-11}
\item \href{https://www.dropbox.com/s/qs58c54zh1kui4d/Ch17InferenceAboutPopulationMean.pdf?dl=0}{B\&M Chapter 17, pages 411-422}
\end{enumerate}

%\showmatmethods


\bibliography{pinp}
\bibliographystyle{jss}



\end{document}

