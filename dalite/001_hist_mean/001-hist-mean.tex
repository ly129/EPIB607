\documentclass[letterpaper,9pt,twocolumn,twoside,printwatermark=false]{pinp}

%% Some pieces required from the pandoc template
\providecommand{\tightlist}{%
  \setlength{\itemsep}{0pt}\setlength{\parskip}{0pt}}

% Use the lineno option to display guide line numbers if required.
% Note that the use of elements such as single-column equations
% may affect the guide line number alignment.

\usepackage[T1]{fontenc}
\usepackage[utf8]{inputenc}

% The geometry package layout settings need to be set here...
\geometry{layoutsize={0.95588\paperwidth,0.98864\paperheight},%
          layouthoffset=0.02206\paperwidth,%
		  layoutvoffset=0.00568\paperheight}

\definecolor{pinpblue}{HTML}{185FAF}  % imagecolorpicker on blue for new R logo
\definecolor{pnasbluetext}{RGB}{101,0,0} %



\title{DALITE Q1 - Histograms, Medians and Means. Due September 12, 2018.}

\author[a]{EPIB607 - Inferential Statistics}

  \affil[a]{Fall 2018, McGill University}

\setcounter{secnumdepth}{5}

% Please give the surname of the lead author for the running footer
\leadauthor{Bhatnagar and Hanley}

% Keywords are not mandatory, but authors are strongly encouraged to provide them. If provided, please include two to five keywords, separated by the pipe symbol, e.g:
 \keywords{  Histogram |  Density Plot |  Mean |  Median |  Mode  }  

\begin{abstract}
This DALITE quiz will cover the basic concepts of histograms, means and
medians.
\end{abstract}

\dates{This version was compiled on \today}
\doi{\url{https://sahirbhatnagar.com/EPIB607/}}

\pinpfootercontents{DALITE Q1 due Sepetember 12, 2018 by 5pm}

\begin{document}

% Optional adjustment to line up main text (after abstract) of first page with line numbers, when using both lineno and twocolumn options.
% You should only change this length when you've finalised the article contents.
\verticaladjustment{-2pt}

\maketitle
\thispagestyle{firststyle}
\ifthenelse{\boolean{shortarticle}}{\ifthenelse{\boolean{singlecolumn}}{\abscontentformatted}{\abscontent}}{}

% If your first paragraph (i.e. with the \dropcap) contains a list environment (quote, quotation, theorem, definition, enumerate, itemize...), the line after the list may have some extra indentation. If this is the case, add \parshape=0 to the end of the list environment.


\section*{Marking}\label{marking}
\addcontentsline{toc}{section}{Marking}

Completion of this DALITE exercise will be availble to us automatically
through the DALITE website. Therefore \textbf{you do not need to hand
anything in}. Marks will be based on the number of correct answers. For
each question you will receive 0.5 marks for getting the correct answer
on the first attempt and an additional 0.5 marks if you stick with the
right answer or switch to the correct answer after seeing someone else's
rationale.

\section{Sign up for DALITE}\label{sign-up-for-dalite}

\textbf{This step only needs to be completed once for the whole
semester}:

\begin{enumerate}
\def\labelenumi{\arabic{enumi}.}
\tightlist
\item
  You can join the EPIB607 group by accessing the unique link:
  \url{https://mydalite.org/en/live/signup/form/NTc4}
\item
  Upon accessing the link, you will be prompted to enter an e-mail
  address. I recommend using the same e-mail address as your DataCamp
  account.
\item
  You never need to remember a username or password to access the DALITE
  platform; access to your assignments is managed through tokens sent to
  your e-mail address. You will be sent a new link everytime a new
  exercise has been posted.
\end{enumerate}

\section{Histograms}\label{histograms}

\subsection{Learning Objectives}\label{learning-objectives}

\begin{enumerate}
\def\labelenumi{\arabic{enumi}.}
\tightlist
\item
  Understand that the distribution of a variable consists of what values
  the variable takes and how often.
\item
  Understand that class intervals should be of equal width; choose
  appropriate class widths to effectively reveal informative patterns in
  the data.
\item
  Understand that the vertical axis of the histogram may be scaled for
  frequency, proportion, or percentage. The choice of vertical scaling
  for any data set does not affect the important features revealed by a
  histogram.
\item
  Be able to describe a graphical display of data by first describing
  the overall pattern and then deviations from that pattern. Describe
  the shape of the overall pattern and identify any gaps in data and
  potential outliers.
\item
  Recognize rough symmetry and clear skewness in the overall pattern of
  a distribution
\end{enumerate}

\subsection{Videos}\label{videos}

\begin{enumerate}
\def\labelenumi{\arabic{enumi}.}
\tightlist
\item
  \href{https://www.learner.org/courses/againstallodds/unitpages/unit03.html}{Against
  All Odds Unit 3}
\end{enumerate}

\vspace*{0.25cm}

\subsection{Required Readings}\label{required-readings}

\begin{enumerate}
\def\labelenumi{\arabic{enumi}.}
\tightlist
\item
  \href{https://www.learner.org/courses/againstallodds/pdfs/AgainstAllOdds_StudentGuide_Unit03.pdf\#page=1}{Against
  All Odds Unit 3, pages 1-6}
\item
  \href{https://serialmentor.com/dataviz/histograms-density-plots.html}{Visualizing
  distributions: Histograms and density plots}
\end{enumerate}

\section{Mean and Median}\label{mean-and-median}

\subsection{Learning Objectives}\label{learning-objectives-1}

\begin{enumerate}
\def\labelenumi{\arabic{enumi}.}
\tightlist
\item
  Understand that graphical descriptions of data are more meaningful
  when supplemented with numerical measures of center.
\item
  Know that the median (midpoint or typical value) and mean (arithmetic
  average) are common measures of center (or location) for a
  distribution.
\item
  Understand the formulas used to calculate the median, mean, and mode.
\item
  Know that the mean and median should be close in symmetric
  distributions and that the mean is pulled toward the long tail of a
  skewed distribution. Know that the mean is a nonresistant measure of
  center because it is strongly influenced by extreme observations and
  that the median is a resistant measure of center.
\item
  Be able to choose an appropriate measure of center in practice.
\end{enumerate}

\subsection{Videos}\label{videos-1}

\begin{enumerate}
\def\labelenumi{\arabic{enumi}.}
\tightlist
\item
  \href{https://www.learner.org/courses/againstallodds/unitpages/unit04.html}{Against
  All Odds Unit 4}
\end{enumerate}

\subsection{Required Readings}\label{required-readings-1}

\begin{enumerate}
\item \href{https://www.learner.org/courses/againstallodds/pdfs/AgainstAllOdds_StudentGuide_Unit04.pdf#page=1}{Against All Odds Unit 4, pages 1-6}
\end{enumerate}

%\showmatmethods


\bibliography{pinp}
\bibliographystyle{jss}



\end{document}

