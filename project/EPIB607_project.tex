\documentclass[letterpaper,12pt,twoside,printwatermark=false]{pinp}

%% Some pieces required from the pandoc template
\providecommand{\tightlist}{%
  \setlength{\itemsep}{0pt}\setlength{\parskip}{0pt}}

% Use the lineno option to display guide line numbers if required.
% Note that the use of elements such as single-column equations
% may affect the guide line number alignment.

\usepackage[T1]{fontenc}
\usepackage[utf8]{inputenc}

% The geometry package layout settings need to be set here...
\geometry{layoutsize={0.95588\paperwidth,0.98864\paperheight},%
          layouthoffset=0.02206\paperwidth,%
		  layoutvoffset=0.00568\paperheight}

\definecolor{pinpblue}{HTML}{185FAF}  % imagecolorpicker on blue for new R logo
\definecolor{pnasbluetext}{RGB}{101,0,0} %



\title{Final Project}

\author[a]{EPIB607 - Inferential Statistics}

  \affil[a]{Fall 2018, McGill University}

\setcounter{secnumdepth}{5}

% Please give the surname of the lead author for the running footer
\leadauthor{Bhatnagar and Hanley}

% Keywords are not mandatory, but authors are strongly encouraged to provide them. If provided, please include two to five keywords, separated by the pipe symbol, e.g:
 \keywords{  Final project  }  

\begin{abstract}
Final project instructions. Due date TBD.
\end{abstract}

\dates{This version was compiled on \today}
\doi{\url{https://sahirbhatnagar.com/EPIB607/}}

\pinpfootercontents{Final Project}

\begin{document}

% Optional adjustment to line up main text (after abstract) of first page with line numbers, when using both lineno and twocolumn options.
% You should only change this length when you've finalised the article contents.
\verticaladjustment{-2pt}

\maketitle
\thispagestyle{firststyle}
\ifthenelse{\boolean{shortarticle}}{\ifthenelse{\boolean{singlecolumn}}{\abscontentformatted}{\abscontent}}{}

% If your first paragraph (i.e. with the \dropcap) contains a list environment (quote, quotation, theorem, definition, enumerate, itemize...), the line after the list may have some extra indentation. If this is the case, add \parshape=0 to the end of the list environment.


\section{Final Project}\label{final-project}

Construct an exercise and solutions suitable for testing or
demonstrating understanding of basic principles of biostatistics as
discussed in this course.

\vspace*{.3in}

Exercises must be based on (i) one to two articles in a scientific
journal or perhaps in the lay press or (ii) a
\textit{publicly available} dataset. The data must not be taken from an
RA project, but must be freely available on the web or another public
source. The article or data should concern some health problem amenable
to statistical investigation. The narrative of the exercise should be
clear and concise. The exercise should comprise 5-7 questions requiring
altogether about one hour for completion. The questions may cover any
part of this course. You must also produce a separate set of model
answers; these should be equally short and to the point.

\vspace*{.3in}

In assessing the quality of your exercise, we shall consider the extent
to which the questions test understanding of important biostatistical
principles in a clear, concise, and unambiguous manner. Credit will also
be given for choice of subject and ingenuity in use of the available
information. The exercise, model answers, and a copy of any published
report(s) or the data on which the exercise is based must be handed in
by the deadline indicated.

\vspace*{.3in}

Projects should be done in groups of 2 to 4 people. Examples final
projects prepared by students in previous years will be posted on
MyCourses. All projects must be uploaded to myCourses. Due date TBD.

\vspace*{.3in}

The upload should consist of at most two files: (1) one file containing
the questions, solutions, and any article(s) on which the questions are
based, and (2) a data-set in text or CSV format, if appropriate.

%\showmatmethods


\bibliography{pinp}
\bibliographystyle{jss}



\end{document}

