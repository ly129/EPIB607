\documentclass[letterpaper,9pt,twoside,printwatermark=false]{pinp}

%% Some pieces required from the pandoc template
\providecommand{\tightlist}{%
  \setlength{\itemsep}{0pt}\setlength{\parskip}{0pt}}

% Use the lineno option to display guide line numbers if required.
% Note that the use of elements such as single-column equations
% may affect the guide line number alignment.

\usepackage[T1]{fontenc}
\usepackage[utf8]{inputenc}

% The geometry package layout settings need to be set here...
\geometry{layoutsize={0.95588\paperwidth,0.98864\paperheight},%
          layouthoffset=0.02206\paperwidth,%
		  layoutvoffset=0.00568\paperheight}

\definecolor{pinpblue}{HTML}{185FAF}  % imagecolorpicker on blue for new R logo
\definecolor{pnasbluetext}{RGB}{101,0,0} %



\title{Exercise 1 - How Deep is the Ocean? September 17, 2018.}

\author[a]{EPIB607 - Inferential Statistics}

  \affil[a]{Fall 2018, McGill University}

\setcounter{secnumdepth}{5}

% Please give the surname of the lead author for the running footer
\leadauthor{Bhatnagar and Hanley}

% Keywords are not mandatory, but authors are strongly encouraged to provide them. If provided, please include two to five keywords, separated by the pipe symbol, e.g:
 \keywords{  Sampling distribution |  Means |  Proportions |  Standard error |  Standard deviation |  Parameter |  Statistic  }  

\begin{abstract}
This in-class exercise will introduce you to sampling distributions for
means and proportions.
\end{abstract}

\dates{This version was compiled on \today}
\doi{\url{https://sahirbhatnagar.com/EPIB607/}}

\pinpfootercontents{Exercise 1: Sampling Distributions for Means and Proportions}

\begin{document}

% Optional adjustment to line up main text (after abstract) of first page with line numbers, when using both lineno and twocolumn options.
% You should only change this length when you've finalised the article contents.
\verticaladjustment{-2pt}

\maketitle
\thispagestyle{firststyle}
\ifthenelse{\boolean{shortarticle}}{\ifthenelse{\boolean{singlecolumn}}{\abscontentformatted}{\abscontent}}{}

% If your first paragraph (i.e. with the \dropcap) contains a list environment (quote, quotation, theorem, definition, enumerate, itemize...), the line after the list may have some extra indentation. If this is the case, add \parshape=0 to the end of the list environment.


\section{What percentage of the world's surface is covered by
water?}\label{what-percentage-of-the-worlds-surface-is-covered-by-water}

The data provided by the
\href{https://topex.ucsd.edu/cgi-bin/get_srtm30.cgi}{Scripps Institution
of Oceanography} can provide an answer, but some work is required on
your part. James Hanley (JH) has randomly sampled \(n=5\) and \(n=20\)
latitudes and longitudes for every student in the class. This document
contains unique latitudes and longitudes for

\begin{ShadedResult}
\begin{verbatim}
[1] "Hanley, James"
\end{verbatim}
\end{ShadedResult}

and was sent in an email (using the
\href{https://cran.r-project.org/package=gmailr}{gmailr} package) as a
pdf attachment to the following address:

\begin{ShadedResult}
\begin{verbatim}
[1] "james.hanley@mcgill.ca"
\end{verbatim}
\end{ShadedResult}

\subsection*{A sample of 5}\label{a-sample-of-5}
\addcontentsline{toc}{subsection}{A sample of 5}

\begin{ShadedResult}
\begin{verbatim}
Longitude.n.5 <- c(116.43,-66.318,166.966,32.036,-118.238)
\end{verbatim}
\end{ShadedResult}\begin{ShadedResult}
\begin{verbatim}
Latitude.n.5 <- c(32.602,-70.789,-51.301,3.274,-32.69)
\end{verbatim}
\end{ShadedResult}

\subsection*{A sample of 20}\label{a-sample-of-20}
\addcontentsline{toc}{subsection}{A sample of 20}

\begin{ShadedResult}
\begin{verbatim}
Longitude.n.20 <- c(57.814,40.026,131.012,-145.304,-130.685,48.011,59.257,111.303,41.137,-61.05,
159.666,-110.195,-87.961,-118.8,99.592,-102.281,159.827,-166.667,-2.171,-6.664)
\end{verbatim}
\end{ShadedResult}\begin{ShadedResult}
\begin{verbatim}
Latitude.n.20 <- c(40.428,59.851,-38.35,-8.862,-26.312,-49.721,-4.02,44.028,-36.308,22.071,
44.765,-18.818,37.185,41.768,-5.255,14.199,61.165,10.874,-27.083,15.792)
\end{verbatim}
\end{ShadedResult}

%\showmatmethods


\bibliography{pinp}
\bibliographystyle{jss}



\end{document}

