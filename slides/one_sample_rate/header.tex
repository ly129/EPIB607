%% header.tex
%%
%% Copyright (C) 2016 - 2017  Dirk Eddelbuettel
%%
%% This file is part of samples-rmarkdown-metropolis repository.
%%
%% samples-rmarkdown-metropolis is free software: you can redistribute it
%% and/or modify it under the terms of the GNU General Public License as
%% published by the Free Software Foundation, either version 2 of the
%% License, or (at your option) any later version.
%%
%% samples-rmarkdown-metropolis is distributed in the hope that it will be
%% useful, but WITHOUT ANY WARRANTY; without even the implied warranty of
%% MERCHANTABILITY or FITNESS FOR A PARTICULAR PURPOSE.  See the GNU General
%% Public License for more details.
%%
%% You should have received a copy of the GNU General Public License along with
%% samples-rmarkdown-metropolis.  If not, see <http://www.gnu.org/licenses/>.

%% If you have the Fira font installed, to actually have it used it 
%% via rmarkdown you need to declare it here 
%\setsansfont[ItalicFont={Fira Sans Light Italic},BoldFont={Fira Sans},BoldItalicFont={Fira Sans Italic}]{Fira Sans Light}
%\setmonofont[BoldFont={Fira Mono Medium}]{Fira Mono}

%% You can set various Metropolis options via \metroset{} here
%\metroset{....}

%% You can redefine colours, mostly by borrowing from Beamer

\usepackage{graphicx}
\graphicspath{ {/home/sahir/Dropbox/jobs/laval/minicours/slides/} }
\usepackage{hyperref, url}
\usepackage[round,sort]{natbib}   % bibliography omit 'round' option if you prefer square brackets
\usepackage[figurename=Fig.]{caption}
\usepackage{subfig}
\usepackage{tikz, pgfplots}
\usetikzlibrary{arrows,shapes.geometric}
\usepackage{animate} %need the animate.sty file 
\usepackage{color, colortbl,xcolor}
\definecolor{lightgray}{RGB}{200,200,200}
\definecolor{myblue}{RGB}{0,89,179}
\setbeamercolor{frametitle}{bg=myblue}
\usepackage{array}
\newcolumntype{L}{>{\centering\arraybackslash}m{3cm}} % used for text wrapping in ctable
\usepackage{ctable}
\usepackage[utf8]{inputenc}
\usepackage{fontenc}
%\usepackage{calrsfs}
%\DeclareMathAlphabet{\pazocal}{OMS}{zplm}{m}{n}
%\newcommand{\La}{\mathcal{L}}
%\newcommand{\Lb}{\pazocal{L}}
\usepackage{pifont}% http://ctan.org/pkg/pifont
\newcommand{\cmark}{\ding{51}}%
\newcommand{\xmark}{\ding{55}}%
\def\widebar#1{\overline{#1}}

\definecolor{whitesmoke}{rgb}{0.96, 0.96, 0.96}
\definecolor{pinkish}{RGB}{255,223,247}
\definecolor{blueish}{RGB}{204,255,255}
\definecolor{lime}{RGB}{0,0,200}

\usepackage[ruled,vlined,noresetcount]{algorithm2e}

\usetikzlibrary{calc}
\usetikzlibrary{backgrounds,intersections,fit}
\tikzset{isometricYXZ/.style={x={(1cm,0cm)}, y={(-1.299cm,-0.75cm)}, z={(0cm,1cm)}}}
\newcommand*\ab{.4}


\usetikzlibrary{shapes.geometric,arrows,shapes.symbols,decorations.pathreplacing,positioning,decorations.markings, matrix, overlay-beamer-styles}
\tikzstyle{startstop} = [rectangle, rounded corners, minimum width=2cm, minimum height=1cm, draw=black, fill=pinkish,text width=11.5cm]
\tikzstyle{startstop2} = [rectangle, rounded corners, minimum width=2cm, minimum height=1cm, draw=black, fill=background,text width=11.5cm]
\tikzstyle{startstop3} = [rectangle, rounded corners, minimum width=3cm, minimum height=1cm, draw=black, fill=beige,text width=11.5cm]
\tikzstyle{startstop4} = [rectangle, rounded corners, minimum width=3cm, minimum height=1cm, draw=black, fill=blueish,text width=11.5cm]
\tikzstyle{io} = [trapezium, trapezium left angle=70, trapezium right angle=110, minimum width=2cm, minimum height=1cm, text centered, draw=black, fill=blue!30,text width=1.5cm]
\tikzstyle{process} = [rectangle, minimum width=1cm, minimum height=1cm, text centered, draw=black, fill=orange!30,text width=2cm]
\tikzstyle{decision} = [diamond, minimum width=2cm, minimum height=1cm, text centered, draw=black, fill=green!30]
\tikzstyle{arrow} = [thick,->,>=stealth]
\tikzstyle{both} = [thick,<->,>=stealth, red]


\tikzset{myshade/.style={minimum size=.4cm,shading=radial,inner color=white,outer color={#1!90!gray}}}
\newcommand\mycirc[1][]{\tikz\node[circle,myshade=#1]{};}
\newcommand\myrect[1][]{\tikz\node[rectangle,myshade=#1]{};}
\newcommand\mystar[1][]{\tikz\node[star,star points=15,star point height=2pt,myshade=#1]{};}
\newcommand\mydiamond[1][]{\tikz\node[diamond,myshade=#1]{};}
\newcommand\myellipse[1][]{\tikz\node[ellipse,myshade=#1]{};}
\newcommand\mykite[1][]{\tikz\node[kite,myshade=#1]{};}
\newcommand\mydart[1][]{\tikz\node[dart,myshade=#1]{};}
\newcommand\mycloud[1][]{\tikz\node[cloud,myshade=#1]{};}

\tikzset{
    ncbar angle/.initial=90,
    ncbar/.style={
        to path=(\tikztostart)
        -- ($(\tikztostart)!#1!\pgfkeysvalueof{/tikz/ncbar angle}:(\tikztotarget)$)
        -- ($(\tikztotarget)!($(\tikztostart)!#1!\pgfkeysvalueof{/tikz/ncbar angle}:(\tikztotarget)$)!\pgfkeysvalueof{/tikz/ncbar angle}:(\tikztostart)$)
        -- (\tikztotarget)
    },
    ncbar/.default=0.5cm,
}

\tikzset{square left brace/.style={ncbar=0.5cm}}
\tikzset{square right brace/.style={ncbar=-0.5cm}}

\tikzset{round left paren/.style={ncbar=0.4cm,out=110,in=-110}}
\tikzset{round right paren/.style={ncbar=0.5cm,out=60,in=-60}}





\pgfdeclareplotmark{students}
{\shade[draw=red!60!black,ball color=red!70] (0pt, 0pt) circle [radius=5pt];} 	
\pgfdeclareplotmark{working}
{\shade[draw=blue!60!black,ball color=blue!70] (0pt, 0pt) circle [radius=5pt];}
\pgfdeclareplotmark{retired}
{\shade[draw=green!60!black,ball color=green!70] (0pt, 0pt) circle [radius=5pt];}



\def\layersep{2.5cm}
\def\Xmean{\skew3\widebar{X}}
\def\Ymean{\widebar{Y}}
\def\xmean{\bar{x}}
\def\ymean{\bar{y}}
\def\dint{\displaystyle\int}
\def\dsum{\displaystyle\sum}


\setbeamercolor{itemize item}{fg=myblue}
\setbeamertemplate{itemize item}[square]

\setbeamertemplate{navigation symbols}{\usebeamercolor[fg]{title in head/foot}\usebeamerfont{title in head/foot}\insertframenumber}
\setbeamertemplate{footline}{}


%\metroset{numbering=fraction, block = fill, titleformat = smallcaps, titleformat plain= smallcaps}
\metroset{block = fill, titleformat = smallcaps, titleformat plain= smallcaps}

\setbeamercolor{background canvas}{bg=white}

%% change fontsize of R code
\let\oldShaded\Shaded
\let\endoldShaded\endShaded
\renewenvironment{Shaded}{\scriptsize\oldShaded}{\endoldShaded}

%% change fontsize of output
\let\oldverbatim\verbatim
\let\endoldverbatim\endverbatim
\renewenvironment{verbatim}{\scriptsize\oldverbatim}{\endoldverbatim}

\newtheorem{proposition}[theorem]{Proposition}
\newtheorem{exercise}[theorem]{Exercise}


%% You also use hyperref, and pick colors 
\hypersetup{colorlinks,citecolor=orange,filecolor=red,linkcolor=brown,urlcolor=blue}

\newcommand {\framedgraphiccaption}[2] {
            \begin{figure}
            \centering
            \includegraphics[width=\textwidth,height=0.8\textheight,keepaspectratio]{#1}
            \caption{#2}
            \end{figure}
}

\newcommand {\framedgraphic}[1] {
            \begin{figure}
            \centering
            \includegraphics[width=\textwidth,height=0.9\textheight,keepaspectratio]{#1}
            \end{figure}
}

\usepackage[cal=cm]{mathalfa}


\newcommand{\tm}[1]{\textrm{{#1}}}
\newcommand{\bx}{\textbf{\emph{x}}}
\newcommand{\by}{\textbf{\emph{y}}}
\newcommand{\bX}{\textbf{X}}
\newcommand{\bW}{\textbf{W}}
\newcommand{\bY}{\textbf{Y}}
\newcommand{\bD}{\textbf{D}}
\newcommand{\bH}{\textbf{H}}
\newcommand{\trans}{\top}
\newcommand{\bXtilde}{\widetilde{\bX}}
\newcommand{\bYtilde}{\widetilde{\bY}}
\newcommand{\bDtilde}{\widetilde{\bD}}
\newcommand{\Xtilde}{\widetilde{X}}
\newcommand{\Ytilde}{\widetilde{Y}}
\newcommand{\Dtilde}{\widetilde{D}}
\newcommand{\bu}{\textbf{u}}
\newcommand{\bU}{\textbf{U}}
\newcommand{\bV}{\textbf{V}}
\newcommand{\bE}{\textbf{E}}
\newcommand{\bb}{\textbf{\emph{b}}}
\newcommand{\bI}{\mathbcal{I}}
\newcommand{\be}{\boldsymbol{\varepsilon}}
\newcommand{\bSigma}{\boldsymbol{\Sigma}}
\newcommand{\bLambda}{\boldsymbol{\Lambda}}
\newcommand{\bTheta}{\boldsymbol{\Theta}}
\newcommand{\balpha}{\boldsymbol{\alpha}}
%\newcommand{\ltwonorm}[1]{\lVert #1 \rVert}
\newcommand{\mb}[1]{\mathbf{#1}}
\newcommand {\bs}{\boldsymbol}
%\newcommand{\norm}[1]{\left\Vert #1 \right\Vert}
\newcommand{\xf}{\mathcal{X}}
\newcommand{\pfrac}[2]{\left( \frac{#1}{#2}\right)}
\newcommand{\e}{{\mathsf E}}
\newcommand{\bt}{\boldsymbol{\theta}}
\newcommand{\bmu}{\boldsymbol{\mu}}
\newcommand{\bbeta}{\boldsymbol{\beta}}
\newcommand{\bbk}{\boldsymbol{\beta}_{(k)}}
\newcommand{\bbkt}{\widetilde{\boldsymbol{\beta}}_{(k)}}
\newcommand{\bgamma}{\boldsymbol{\gamma}}
\newcommand{\btheta}{\boldsymbol{\theta}}
\newcommand{\bPhi}{\boldsymbol{\Phi}}
\newcommand{\bPsi}{\boldsymbol{\Psi}}
\DeclareMathOperator*{\argmin}{arg\,min}
\DeclareMathOperator*{\argmax}{arg\,max}
\DeclareMathOperator{\diag}{diag} % operator and subscript


\definecolor{myblues}{cmyk}{10,0,0,0}
\definecolor{darktangerine}{rgb}{1.0, 0.66, 0.07}
\definecolor{deepink}{RGB}{255,20,147}


%\usepackage{amsthm}
\setbeamertemplate{theorems}[numbered]

%% when rendered with rmarkdown, somehow the unicode char for the dot
%% disappears so we redefine it here -- that is an older comments, seems font-specific
%\renewcommand{\textbullet}{$\cdot$}
%\renewcommand{\itemBullet}{▸}   % unicode U+25b8 'black right pointing small triangle'

%% The institute macro puts a small line for affiliation at the bottom
\institute{le 8 février, 2018} 

%% We can also place a logo
\titlegraphic{\hfill\includegraphics[height=1cm]{ulaval-logo.jpg}}

%%% Local Variables:
%%% mode: latex
%%% TeX-master: t
%%% End:
