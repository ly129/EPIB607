\documentclass[]{article}
\usepackage[pagebackref=true,bookmarks]{hyperref}
                             % Neat package to turn href, ref, cite, gloss entries
                             % into hyperlinks in the dvi file.
                             % Make sure this is the last package loaded.
                             % Use with dvips option to get hyperlinks to work in ps and pdf
                             % files.  Unfortunately, then they don't work in the dvi file!
                             % Use without the dvips option to get the links to work in the dvi file.

                             % Note:  \floatstyle{ruled} don't work properly; so change to plain.
                             % Not as pretty, but functional...
                             % The bookmarks option sets up proper bookmarks in the pdf file :)
\hypersetup{
    unicode=false,          
    pdftoolbar=true,        
    pdfmenubar=true,        
    pdffitwindow=false,     % window fit to page when opened
    pdfstartview={FitH},    % fits the width of the page to the window
    pdftitle={EPIB 607 - Fall 2015 Course Outline},    % title
    pdfauthor={Paramita Saha Chaudhuri},     % author
    pdfsubject={Subject},   % subject of the document
    pdfcreator={Sahir Rai Bhatnagar},   % creator of the document
    pdfproducer={Sahir Rai Bhatnagar}, % producer of the document
    pdfkeywords={}, % list of keywords
    pdfnewwindow=true,      % links in new window
    colorlinks=true,       % false: boxed links; true: colored links
    linkcolor=red,          % color of internal links (change box color with linkbordercolor)
    citecolor=blue,        % color of links to bibliography
    filecolor=black,      % color of file links
    urlcolor=blue           % color of external links
}


\usepackage{color}

%opening
\title{EPIB 607 - Fall 2018\\Principles of Inferential Statistics in Medicine\\\url{https://sahirbhatnagar.com/EPIB607}}
\author{4 credits\\Mondays 11:30-13:30 and Thursdays, 8:30-10:30\\
Location: McMed 1034}

\date{}

\begin{document}

\maketitle

\section{Instructor}
Sahir Rai Bhatnagar (\url{https://sahirbhatnagar.com/}) \\
Department of Epidemiology, Biostatistics, and Occupational Health\\
Department of Diagnostic Radiology\\
McGill University\\
\href{mailto:sahir.bhatnagar@mcgill.ca}{\color{blue}{sahir.bhatnagar@mcgill.ca}}\\
Office hours: Tuesdays in Purvis Hall 37 \textit{by appointment}

\vspace{0.40cm}

\noindent TA: Kody Crowell, Guanbo Wang, Himasara Marasinghe  \\
Office hours: TBA



\section{Prerequisites}
Basic understanding of exponentials, logs, histograms, graphs, mean, median, mode, standard deviation. Enrollment in the Epidemiology or Public Health program at McGill University.

\section{Objectives}
The aim of this course is to provide students with basic principles of statistical inference so that they can:

\begin{itemize}
\item Visualize/Analyze/Interpret data using statistical methods with the \texttt{R} statistical software program.
\item Understand the statistical results in a scientific paper.
\item Apply statistical methods in their own research.
\item Use the methods learned in this course as a foundation for more advanced biostatistics courses.
\end{itemize}


\section{Audience}
The principal audience is researchers in the natural and social sciences who haven't had an introductory course in statistics (or did have one a long time ago). This audience accepts that statistics has penetrated the life sciences pervasively and is required knowledge for both doing research and understanding scientific papers.

\section{Teaching strategy}
\textbf{This course will follow the Flipped Classroom model}: Here, students are expected to have engaged with the material before coming to class (based on very precise pre-class instructions). The students will then be expected to answer a series of conceptual multiple choice questions using the DALITE online platform (\url{https://mydalite.org/}, \url{https://www.youtube.com/watch?v=0tJVVy2ay7c}). This allows the instructor to delegate the delivery of basic content and definitions to textbooks and videos, and enforces the idea that students cannot be simply passive recipients of information. This approach then allows the professor to focus valuable class time on nurturing efficient discussions surrounding the ideas within the content, guiding interactive exploration of typical misconceptions, and promoting collaborative problem solving with peers.\\

\textbf{A focus on computation}: Classic introductory statistics textbooks were written during a time when computers were still in their infancy. As such, even the newer editions heavily rely on by-hand computations such as looking up tables for tail probabilities. We take a modern approach and introduce computational methods in statistics with the statistical software program \texttt{R}.


\section{Tutorials from DataCamp}
This class is supported by DataCamp, which will allow you to learn \texttt{R} through a combination of short expert videos and hands-on-the-keyboard exercises. You will be asked to complete some of the courses in DataCamp for background reading or for assignments. You can sign up for a free account at this \href{https://www.datacamp.com/groups/shared_links/4c7d78a632b557dfdd6618b3e8fac09495571fec}{link}. Note: you are required to sign up with a @mail.mcgill.ca or @mcgill.ca email address.

\section{Content}
Course structure will consist of elaborating selected topics from the book Baldi \& Moore : ``The Practice of Statistics in the Life Sciences'', 3$^{rd}$ edition. You will also be asked to watched the accompanying \href{https://www.learner.org/courses/againstallodds/unitpages/index.html}{Against all odds video series} by Annenberg Learner. We will also cover more advanced material not covered in the textbook, for which class notes will be made available. 

\subsection{Descriptive Statistics}
\begin{itemize}
\item Histograms, density plots, measures of center, boxplots, standard deviation
\item Data visualization (aesthetics, visual cues, coordinate systems, scales, facets and layers)
\item Choosing color palettes: Cynthia Brewer palettes, perceptually uniform palettes, color blind friendly palettes. 
\item Tidy data
\end{itemize}

\subsection{Sampling Distributions}
\begin{itemize}
\item Parameters and statistics
\item Standard error of the mean
\item Normal (Gaussian) distribution
\item Central Limit Theorem
\item Confidence intervals
\item Bootstrap for sampling distributions and confidence intervals
\end{itemize}

\subsection{One Sample Inference}
\begin{itemize}
\item Inference about a population mean
\item P values, power, and sample size considerations
\item Inference about a population proportion
\item Inference about a population rate
\end{itemize}


\subsection{Regression}
\begin{itemize}
\item Linear regression for means, difference in means, ratio of means
\item Poisson regression for rates, rate differences, rate ratios
\item Logistic regression for odds ratios and risk ratios
\end{itemize}




\section{References}
\subsection{Optional text}
Baldi B and Moore D S. \textit{The Practice of Statistics in the Life Sciences}, 3${rd}$ edition. Freeman and Company.\\

\subsection{Course notes}
These are available as a PDF document on MyCourses to download.


\section{Equipment}
Hand calculators (with square root, log, and exponential function) are required.  


\section{Software}
The \texttt{R} software package will be introduced and used for in-class illustrations. \texttt{R} is available under GPL (free) at \href{http://cran.r-project.org/}{\color{blue}{http://cran.r-project.org/}}. \\

%\vspace{0.51cm}
\noindent
It is recommended to use the RStudio interactive development environment (IDE) which can be downloaded for free at \href{http://www.rstudio.com/}{\color{blue}{http://www.rstudio.com/}}.\\ 
\textbf{Note:} to use RStudio, you must first download the  \texttt{R}
software package at the link provided above.\\


\section{Evaluation}
\noindent
\begin{tabular}{lc}
Assignments (submit using MyCourses) & 40\%\\
Midterm exam (Formula sheet) & 15\%  \\
DALITE & 15\%\\
Project & 10\% \\
Final exam (Formula sheet) & 20\%
\end{tabular}
\newline

\vspace{.51cm}

\noindent
The final grade will consist of a letter grade.

\section{Note on academic integrity}

McGill University values academic integrity. Therefore all students must understand the meaning and consequences of cheating, plagiarism and other academic offences under the Code of Student Conduct and Disciplinary Procedures (see \href{http://www.mcgill.ca/integrity/}{\color{blue}{http://www.mcgill.ca/integrity/}} for more information).


\end{document}
