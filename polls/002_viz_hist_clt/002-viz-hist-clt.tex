\documentclass[letterpaper,10pt,twoside,printwatermark=false]{pinp}

%% Some pieces required from the pandoc template
\providecommand{\tightlist}{%
  \setlength{\itemsep}{0pt}\setlength{\parskip}{0pt}}

% Use the lineno option to display guide line numbers if required.
% Note that the use of elements such as single-column equations
% may affect the guide line number alignment.

\usepackage[T1]{fontenc}
\usepackage[utf8]{inputenc}

% The geometry package layout settings need to be set here...
\geometry{layoutsize={0.95588\paperwidth,0.98864\paperheight},%
          layouthoffset=0.02206\paperwidth,%
		  layoutvoffset=0.00568\paperheight}

\definecolor{pinpblue}{HTML}{185FAF}  % imagecolorpicker on blue for new R logo
\definecolor{pnasbluetext}{RGB}{101,0,0} %



\title{Poll 2 - Data Visualization, Histograms, Measures of Center, CLT}

\author[a]{EPIB607 - Inferential Statistics}

  \affil[a]{Fall 2018, McGill University}

\setcounter{secnumdepth}{5}

% Please give the surname of the lead author for the running footer
\leadauthor{Bhatnagar and Hanley}

% Keywords are not mandatory, but authors are strongly encouraged to provide them. If provided, please include two to five keywords, separated by the pipe symbol, e.g:
 \keywords{  Descriptive stats |  Confidence Interval |  p-value |  Gaussian distribution |  CLT |  Simple linear regression  }  

\begin{abstract}
This live poll was conducted on September 24, 2018. This is meant to
review some of the main concepts we have learned up untill now. Correct
answers are indicated by check marks. Number of votes and number of
participants are indicated in the figure legend. For some questions,
several selections were allowed.
\end{abstract}

\dates{This version was compiled on \today}
\doi{\url{https://sahirbhatnagar.com/EPIB607/}}

\pinpfootercontents{Poll 2 - September 24, 2018}

\begin{document}

% Optional adjustment to line up main text (after abstract) of first page with line numbers, when using both lineno and twocolumn options.
% You should only change this length when you've finalised the article contents.
\verticaladjustment{-2pt}

\maketitle
\thispagestyle{firststyle}
\ifthenelse{\boolean{shortarticle}}{\ifthenelse{\boolean{singlecolumn}}{\abscontentformatted}{\abscontent}}{}

% If your first paragraph (i.e. with the \dropcap) contains a list environment (quote, quotation, theorem, definition, enumerate, itemize...), the line after the list may have some extra indentation. If this is the case, add \parshape=0 to the end of the list environment.


\section{Which of the following aesthetics can be used to represent
continuous data (select all that
apply)?}\label{which-of-the-following-aesthetics-can-be-used-to-represent-continuous-data-select-all-that-apply}

\begin{enumerate}
\def\labelenumi{\arabic{enumi}.}
\tightlist
\item
  position \textbf{(Correct)}
\item
  shape
\item
  size \textbf{(Correct)}
\item
  color \textbf{(Correct)}
\item
  line width \textbf{(Correct)}
\item
  line type
\end{enumerate}

\section{Which of the following statement characterizes a scale (select
all that
apply)?}\label{which-of-the-following-statement-characterizes-a-scale-select-all-that-apply}

\begin{enumerate}
\def\labelenumi{\arabic{enumi}.}
\tightlist
\item
  Defines a unique mapping between data \& aesthetics \textbf{(Correct)}
\item
  Must be 1-to-many between data \& aesthetics
\item
  Must be 1-to-1 between data \& aesthetics \textbf{(Correct)}
\end{enumerate}

\section{Which color palette would you use to distinguish groups of data
from each other (select all that
apply)?}\label{which-color-palette-would-you-use-to-distinguish-groups-of-data-from-each-other-select-all-that-apply}

\begin{enumerate}
\def\labelenumi{\arabic{enumi}.}
\tightlist
\item
  Sequential (Brewer palette)
\item
  Qualitative (Brewer palette) \textbf{(Correct)}
\item
  Diverging (Brewer palette)
\item
  Viridis
\end{enumerate}

\section{Which color palette would you use to represent continuous data
values, such as income, temperature, or speed (select all that
apply)?}\label{which-color-palette-would-you-use-to-represent-continuous-data-values-such-as-income-temperature-or-speed-select-all-that-apply}

\begin{enumerate}
\def\labelenumi{\arabic{enumi}.}
\tightlist
\item
  Sequential (Brewer palette) \textbf{(Correct)}
\item
  Qualitative (Brewer palette)
\item
  Diverging (Brewer palette) \textbf{(Correct)}
\item
  Viridis \textbf{(Correct)}
\end{enumerate}

\section{The area under a frequency histogram must equal to
1}\label{the-area-under-a-frequency-histogram-must-equal-to-1}

\begin{enumerate}
\def\labelenumi{\arabic{enumi}.}
\tightlist
\item
  TRUE
\item
  FALSE \textbf{(Correct)}
\end{enumerate}

\section{A boxplot can show whether a data set
is}\label{a-boxplot-can-show-whether-a-data-set-is}

\begin{enumerate}
\def\labelenumi{\arabic{enumi}.}
\tightlist
\item
  symmetric
\item
  skewed
\item
  symmetric and skewed \textbf{(Correct)}
\end{enumerate}

\section{If one side of the box is longer than the other, it means that
side contains more
data.}\label{if-one-side-of-the-box-is-longer-than-the-other-it-means-that-side-contains-more-data.}

\begin{enumerate}
\def\labelenumi{\arabic{enumi}.}
\tightlist
\item
  TRUE
\item
  FALSE \textbf{(Correct)}
\end{enumerate}

\section{To construct a boxplot, we need (select all that
apply)}\label{to-construct-a-boxplot-we-need-select-all-that-apply}

\begin{enumerate}
\def\labelenumi{\arabic{enumi}.}
\tightlist
\item
  Interquartile range
\item
  Minimum \textbf{(Correct)}
\item
  Maximum \textbf{(Correct)}
\item
  Standard deviation
\item
  Mean
\item
  Median \textbf{(Correct)}
\item
  Mode
\item
  Skewness
\item
  1st quartile (Q1) \textbf{(Correct)}
\item
  3rd quartile (Q3) \textbf{(Correct)}
\end{enumerate}

\section{In a distribution with a long left tail, the mean
is}\label{in-a-distribution-with-a-long-left-tail-the-mean-is}

\begin{enumerate}
\def\labelenumi{\arabic{enumi}.}
\tightlist
\item
  greater than the median
\item
  less than the median \textbf{(Correct)}
\item
  equal to the median
\end{enumerate}

\section{Which of the following are true concerning a parameter (select
all that
apply)?}\label{which-of-the-following-are-true-concerning-a-parameter-select-all-that-apply}

\begin{enumerate}
\def\labelenumi{\arabic{enumi}.}
\tightlist
\item
  A numerical constant \textbf{(Correct)}
\item
  Pertains to a population \textbf{(Correct)}
\item
  Is unknown \textbf{(Correct)}
\item
  is known
\item
  Is a statistic
\item
  \(\bar{y}\) and \(p\) (the sample proportion) are parameters
\end{enumerate}

\section{The standard error of the mean (SEM) describes (select all that
apply)}\label{the-standard-error-of-the-mean-sem-describes-select-all-that-apply}

\begin{enumerate}
\def\labelenumi{\arabic{enumi}.}
\tightlist
\item
  How far \(\bar{y}\) could typically deviate from \(\mu\)
  \textbf{(Correct)}
\item
  How far an individual \(y\) typically deviates from \(\mu\) or from
  \(\bar{y}\)
\end{enumerate}

\section{The Central Limit Theorem states that (select all that
apply)}\label{the-central-limit-theorem-states-that-select-all-that-apply}

\begin{enumerate}
\def\labelenumi{\arabic{enumi}.}
\tightlist
\item
  \(\bar{y} \sim \mathcal{N}(\mu, \sigma / \sqrt{n})\)
\item
  \(\bar{y} \sim \mathcal{N}(\mu, \sigma / \sqrt{n})\) for large enough
  \(n\)
\item
  \(\bar{y} \sim \mathcal{N}(\mu, \sigma / \sqrt{n})\) for large enough
  \(n\) and finite variance \textbf{(Correct)}
\item
  The sampling distribution of \(\bar{y}\) is, for a large enough \(n\)
  and finite variance, close to Gaussian in shape no matter what the
  shape of the distribution of individual \(Y\) values
  \textbf{(Correct)}
\end{enumerate}

%\showmatmethods


\bibliography{pinp}
\bibliographystyle{jss}



\end{document}

