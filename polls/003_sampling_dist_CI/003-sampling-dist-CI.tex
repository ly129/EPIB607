\documentclass[letterpaper,10pt,twoside,printwatermark=false]{pinp}

%% Some pieces required from the pandoc template
\providecommand{\tightlist}{%
  \setlength{\itemsep}{0pt}\setlength{\parskip}{0pt}}

% Use the lineno option to display guide line numbers if required.
% Note that the use of elements such as single-column equations
% may affect the guide line number alignment.

\usepackage[T1]{fontenc}
\usepackage[utf8]{inputenc}

% The geometry package layout settings need to be set here...
\geometry{layoutsize={0.95588\paperwidth,0.98864\paperheight},%
          layouthoffset=0.02206\paperwidth,%
		  layoutvoffset=0.00568\paperheight}

\definecolor{pinpblue}{HTML}{185FAF}  % imagecolorpicker on blue for new R logo
\definecolor{pnasbluetext}{RGB}{101,0,0} %



\title{Poll 3 - Sampling Distribution, Confidence Intervals}

\author[a]{EPIB607 - Inferential Statistics}

  \affil[a]{Fall 2018, McGill University}

\setcounter{secnumdepth}{5}

% Please give the surname of the lead author for the running footer
\leadauthor{Bhatnagar and Hanley}

% Keywords are not mandatory, but authors are strongly encouraged to provide them. If provided, please include two to five keywords, separated by the pipe symbol, e.g:
 \keywords{  Sampling distribution |  Confidence interval |  qnorm |  pnorm  }  

\begin{abstract}
This live poll was conducted on October 4, 2018.
\end{abstract}

\dates{This version was compiled on \today}
\doi{\url{https://sahirbhatnagar.com/EPIB607/}}

\pinpfootercontents{Poll 3 - October 4, 2018}

\begin{document}

% Optional adjustment to line up main text (after abstract) of first page with line numbers, when using both lineno and twocolumn options.
% You should only change this length when you've finalised the article contents.
\verticaladjustment{-2pt}

\maketitle
\thispagestyle{firststyle}
\ifthenelse{\boolean{shortarticle}}{\ifthenelse{\boolean{singlecolumn}}{\abscontentformatted}{\abscontent}}{}

% If your first paragraph (i.e. with the \dropcap) contains a list environment (quote, quotation, theorem, definition, enumerate, itemize...), the line after the list may have some extra indentation. If this is the case, add \parshape=0 to the end of the list environment.


\section{Which R function would we use to answer the following question:
what is the probability of seeing an IQ score as extreme as 130 (select
all that
apply)?}\label{which-r-function-would-we-use-to-answer-the-following-question-what-is-the-probability-of-seeing-an-iq-score-as-extreme-as-130-select-all-that-apply}

\begin{enumerate}
\def\labelenumi{\arabic{enumi}.}
\tightlist
\item
  pnorm \textbf{(Correct)}
\item
  qnorm
\item
  dnorm
\item
  rnorm
\end{enumerate}

\section{Which R function would we use to answer the following question:
What is the 75th percentile of the IQ scores distribution (select all
that
apply)?}\label{which-r-function-would-we-use-to-answer-the-following-question-what-is-the-75th-percentile-of-the-iq-scores-distribution-select-all-that-apply}

\begin{enumerate}
\def\labelenumi{\arabic{enumi}.}
\tightlist
\item
  pnorm
\item
  qnorm \textbf{(Correct)}
\item
  dnorm
\item
  rnorm
\end{enumerate}

\section{\texorpdfstring{The population SD is unknown and is denoted by
\(\sigma\). In an SRS of size \(n\), the standard deviation of that
sample
is}{The population SD is unknown and is denoted by \textbackslash{}sigma. In an SRS of size n, the standard deviation of that sample is}}\label{the-population-sd-is-unknown-and-is-denoted-by-sigma.-in-an-srs-of-size-n-the-standard-deviation-of-that-sample-is}

\begin{enumerate}
\def\labelenumi{\arabic{enumi}.}
\tightlist
\item
  the sample standard deviation \textbf{(Correct)}
\item
  the sample standard deviation divided by \(\sqrt{n}\)
\item
  \(\sigma / \sqrt{n}\)
\end{enumerate}

\section{\texorpdfstring{The sampling distribution of \(\bar{y}\)
(select all that
apply):}{The sampling distribution of \textbackslash{}bar\{y\} (select all that apply):}}\label{the-sampling-distribution-of-bary-select-all-that-apply}

\begin{enumerate}
\def\labelenumi{\arabic{enumi}.}
\tightlist
\item
  describes how the statistic \(\bar{y}\) varies in all possible SRSs of
  the same size from the same population \textbf{(Correct)}
\item
  is Normally distributed
\item
  is Normally distributed only if the population distribution is Normal
\item
  is usually unknown \textbf{(Correct)}
\item
  is centered around the population mean \(\mu\) \textbf{(Correct)}
\item
  has an SD greater than the population SD for an SRS of size \(n\)
\end{enumerate}

\section{The central limit theorem applies to one random
sample}\label{the-central-limit-theorem-applies-to-one-random-sample}

\begin{enumerate}
\def\labelenumi{\arabic{enumi}.}
\tightlist
\item
  TRUE
\item
  FALSE \textbf{(Correct)}
\end{enumerate}

\section{\texorpdfstring{A 95\% confidence interval for the mean \(\mu\)
(select all that
apply)}{A 95\% confidence interval for the mean \textbackslash{}mu (select all that apply)}}\label{a-95-confidence-interval-for-the-mean-mu-select-all-that-apply}

\begin{enumerate}
\def\labelenumi{\arabic{enumi}.}
\tightlist
\item
  is a random quantity \textbf{(Correct)}
\item
  tells us that in the long run, 95\% of your intervals will contain the
  mean \(\mu\) \textbf{(Correct)}
\item
  tells us that in the long run, 95\% of your intervals will contain the
  sample mean \(\bar{y}\)
\end{enumerate}

\section{\texorpdfstring{Which of the following assumptions are needed
to be able to use a formula of the form
\(\bar{y} \pm z^\star (\sigma / \sqrt{n})\) (select all that
apply)}{Which of the following assumptions are needed to be able to use a formula of the form \textbackslash{}bar\{y\} \textbackslash{}pm z\^{}\textbackslash{}star (\textbackslash{}sigma / \textbackslash{}sqrt\{n\}) (select all that apply)}}\label{which-of-the-following-assumptions-are-needed-to-be-able-to-use-a-formula-of-the-form-bary-pm-zstar-sigma-sqrtn-select-all-that-apply}

\begin{enumerate}
\def\labelenumi{\arabic{enumi}.}
\tightlist
\item
  the population distribution must be normal
\item
  the CLT has `kicked in' \textbf{(Correct)}
\item
  we have an SRS of size \(n\) from the population of interest
  \textbf{(Correct)}
\end{enumerate}

\section{\texorpdfstring{A 95\% CI for the mean \(\mu\) is given by
\(\bar{y} \pm z^\star (\sigma / \sqrt{n})\). This can be calculated in R
using the following command (select all that
apply)}{A 95\% CI for the mean \textbackslash{}mu is given by \textbackslash{}bar\{y\} \textbackslash{}pm z\^{}\textbackslash{}star (\textbackslash{}sigma / \textbackslash{}sqrt\{n\}). This can be calculated in R using the following command (select all that apply)}}\label{a-95-ci-for-the-mean-mu-is-given-by-bary-pm-zstar-sigma-sqrtn.-this-can-be-calculated-in-r-using-the-following-command-select-all-that-apply}

\begin{enumerate}
\def\labelenumi{\arabic{enumi}.}
\tightlist
\item
  \(\bar{y}\) + qnorm(p = c(0.025, 0.975)) \((\sigma / \sqrt{n})\)
  \textbf{(Correct)}
\item
  \(\bar{y}\) + qnorm(p = c(0.025, 0.975), mean = \(\bar{y}\), sd =
  \(\sigma\)) \((\sigma / \sqrt{n})\)
\item
  \(\bar{y}\) + qnorm(p = c(0.025, 0.975), mean = \(\bar{y}\), sd =
  \(\sigma / \sqrt{n}\))
\item
  \(\bar{y}\) + qnorm(p = c(0.025, 0.975), mean = 0, sd =
  \(\sigma / \sqrt{n}\))
\item
  qnorm(p = c(0.025, 0.975), mean = \(\bar{y}\), sd =
  \(\sigma / \sqrt{n}\)) \textbf{(Correct)}
\end{enumerate}

%\showmatmethods


\bibliography{pinp}
\bibliographystyle{jss}



\end{document}

