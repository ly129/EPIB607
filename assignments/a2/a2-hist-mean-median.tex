\documentclass[letterpaper,9pt,twocolumn,twoside,printwatermark=false]{pinp}

%% Some pieces required from the pandoc template
\providecommand{\tightlist}{%
  \setlength{\itemsep}{0pt}\setlength{\parskip}{0pt}}

% Use the lineno option to display guide line numbers if required.
% Note that the use of elements such as single-column equations
% may affect the guide line number alignment.

\usepackage[T1]{fontenc}
\usepackage[utf8]{inputenc}

% The geometry package layout settings need to be set here...
\geometry{layoutsize={0.95588\paperwidth,0.98864\paperheight},%
          layouthoffset=0.02206\paperwidth,%
		  layoutvoffset=0.00568\paperheight}

\definecolor{pinpblue}{HTML}{185FAF}  % imagecolorpicker on blue for new R logo
\definecolor{pnasbluetext}{RGB}{101,0,0} %



\title{Assignment 2 - Histograms, Medians, Means, Boxplots and Standard
Deviation. Due September 21, 11:59pm 2018}

\author[a]{EPIB607 - Inferential Statistics}

  \affil[a]{Fall 2018, McGill University}

\setcounter{secnumdepth}{5}

% Please give the surname of the lead author for the running footer
\leadauthor{Bhatnagar and Hanley}

% Keywords are not mandatory, but authors are strongly encouraged to provide them. If provided, please include two to five keywords, separated by the pipe symbol, e.g:
 \keywords{  Histograms |  Means |  Medians |  Boxplots |  Standard Deviation |  Summary statistics |  mosaic package  }  

\begin{abstract}
The first step in understanding data is to hear what the data say, to
``let the data speak for themselves''. Numbers speak clearly only when
we help them speak by organizing, displaying, and summarizing. In this
assignment you will explore how to visualize your data and summarize it
using summary statistics. All graphs and calculations are to be
completed in an R Markdown document using the provided template. Please
submit both the compiled HTML report and the source file (.Rmd) to
myCourses by September 21, 2018, 11:59pm. Both HTML and .Rmd files
should be saved as `IDnumber\_LastName\_FirstName\_EPIB607\_A2'.
\end{abstract}

\dates{This version was compiled on \today}
\doi{\url{https://sahirbhatnagar.com/EPIB607/}}

\pinpfootercontents{Assignment 2 due Sepetember 21, 2018 by 11:59pm}

\begin{document}

% Optional adjustment to line up main text (after abstract) of first page with line numbers, when using both lineno and twocolumn options.
% You should only change this length when you've finalised the article contents.
\verticaladjustment{-2pt}

\maketitle
\thispagestyle{firststyle}
\ifthenelse{\boolean{shortarticle}}{\ifthenelse{\boolean{singlecolumn}}{\abscontentformatted}{\abscontent}}{}

% If your first paragraph (i.e. with the \dropcap) contains a list environment (quote, quotation, theorem, definition, enumerate, itemize...), the line after the list may have some extra indentation. If this is the case, add \parshape=0 to the end of the list environment.


\section*{Template}\label{template}
\addcontentsline{toc}{section}{Template}

The \texttt{.Rmd} template for Assignment 2 is available
\href{sahirbhatnagar.com}{here}

\section{\texorpdfstring{The \texttt{mosaic} package
(optional)}{The mosaic package (optional)}}\label{the-mosaic-package-optional}

The \texttt{mosaic} package provides a consistent and user-friendly
interface for descriptive statistics, plots and inference. You may find
it useful to complete an interactive tutorial on its plotting functions.
(note: this is optional and will not be counted for any marks). First
install the following packages:

\begin{Shaded}
\begin{Highlighting}[]
\KeywordTok{install.packages}\NormalTok{(}\KeywordTok{c}\NormalTok{(}\StringTok{"learnr"}\NormalTok{,}\StringTok{"mosaic"}\NormalTok{), }
                 \DataTypeTok{dependencies =} \OtherTok{TRUE}\NormalTok{)}
\end{Highlighting}
\end{Shaded}

Then, from RStudio, run the following command which will open a new page
in your web broswer:

\begin{Shaded}
\begin{Highlighting}[]
\NormalTok{learnr}\OperatorTok{::}\KeywordTok{run_tutorial}\NormalTok{(}\StringTok{"introduction"}\NormalTok{, }
                     \DataTypeTok{package =} \StringTok{"ggformula"}\NormalTok{)}
\end{Highlighting}
\end{Shaded}

A more advanced tutorial on customizing your plots is available also:

\begin{Shaded}
\begin{Highlighting}[]
\NormalTok{learnr}\OperatorTok{::}\KeywordTok{run_tutorial}\NormalTok{(}\StringTok{"refining"}\NormalTok{, }
                     \DataTypeTok{package =} \StringTok{"ggformula"}\NormalTok{)}
\end{Highlighting}
\end{Shaded}

\section{Age-structures of Populations, then and
now}\label{age-structures-of-populations-then-and-now}

\textbf{Please try to complete this step before class on Thursday
September 6}:

\begin{enumerate}
\def\labelenumi{\arabic{enumi}.}
\tightlist
\item
  Sign up for a free DataCamp account at
  \href{https://www.datacamp.com/groups/shared_links/4c7d78a632b557dfdd6618b3e8fac09495571fec}{this
  link}. Note: you are required to sign up with a
  \texttt{@mail.mcgill.ca} or \texttt{@mcgill.ca} email address.\\
\item
  Sign in to \url{https://github.mcgill.ca/} using your McGill email
  address and corresponding password.
\end{enumerate}

\begin{Shaded}
\begin{Highlighting}[]
\NormalTok{dat <-}\StringTok{ }\KeywordTok{read.table}\NormalTok{(}\StringTok{"~/git_repositories/epib607/data/age_sex_frequencies_ireland.txt"}\NormalTok{, }\DataTypeTok{header =} \OtherTok{TRUE}\NormalTok{)}
\KeywordTok{library}\NormalTok{(dplyr)}
\NormalTok{dat <-}\StringTok{ }\NormalTok{dat }\OperatorTok\StringTok{ }\KeywordTok{mutate}\NormalTok{(}\DataTypeTok{Male =} \KeywordTok{case_when}\NormalTok{(Male }\OperatorTok{==}\StringTok{ }\DecValTok{1} \OperatorTok{~}\StringTok{ "Male"}\NormalTok{, Male }\OperatorTok{==}\StringTok{ }\DecValTok{0} \OperatorTok{~}\StringTok{ "Female"}\NormalTok{))}
\KeywordTok{colnames}\NormalTok{(dat) <-}\StringTok{ }\KeywordTok{c}\NormalTok{(}\StringTok{"Gender"}\NormalTok{, }\StringTok{"Age"}\NormalTok{, }\StringTok{"Freq"}\NormalTok{)}
\KeywordTok{write.csv}\NormalTok{(dat, }\DataTypeTok{file =} \StringTok{"data/age_sex_frequencies_ireland.csv"}\NormalTok{, }\DataTypeTok{quote =} \OtherTok{FALSE}\NormalTok{, }\DataTypeTok{row.names =} \OtherTok{FALSE}\NormalTok{)}


\KeywordTok{library}\NormalTok{(rmarkdown)}
\KeywordTok{draft}\NormalTok{(}\StringTok{"mypaper.Rmd"}\NormalTok{, }\DataTypeTok{template=}\StringTok{"pdf"}\NormalTok{, }\DataTypeTok{package=}\StringTok{"pinp"}\NormalTok{, }\DataTypeTok{edit=}\OtherTok{TRUE}\NormalTok{)}
\end{Highlighting}
\end{Shaded}

\section{Install Git}\label{install-git}

You need to first install the \href{https://git-scm.com/}{git} version
control system on your system. Follow Chapter 1: Installing Git
\href{https://plot.ly/r/github-getting-started-for-data-scientists/\#chapter-1-installing-git}{at
this link} for step-by-step installation instructions with screenshots.

\section{Install R and RStudio}\label{install-r-and-rstudio}

This short course will guide you through installing both
\href{https://cran.r-project.org/}{R} and
\href{https://www.rstudio.com/products/rstudio/download/preview/}{RStudio}.
RStudio is a software application that facilitates how you interact with
\texttt{R}. Click on the link the in the figure caption join the course.

\begin{figure}[H]
  \begin{center}
    \includegraphics[width=1in, height=1in]{../../images/rstudio_ide.png} 
  \end{center}
  \caption{\href{https://www.datacamp.com/courses/working-with-the-rstudio-ide-part-1}{Working with the RStudio IDE (Part 1)}}\label{fig}
\end{figure}

\section{Introduction to R}\label{introduction-to-r}

In this course you will get a hands-on introduction to the basic
commands in R. With the knowledge gained in this course, you will be
ready to perform a data analysis. Click on the link the in the figure
caption join the course.

\begin{figure}[H]
  \begin{center}
    \includegraphics[width=1in, height=1in]{../../images/intro_r.png} 
  \end{center}
  \caption{\href{https://www.datacamp.com/courses/free-introduction-to-r}{Introduction to R}}\label{fig}
\end{figure}

\section{R Markdown}\label{r-markdown}

You will learn how to create reproducible reports using R and Markdown.
All assignments for this course must be submitted in this format. Click
on the link the in the figure caption join the course.

\begin{figure}[H]
  \begin{center}
    \includegraphics[width=1in, height=1in]{../../images/rmarkdown_r.png} 
  \end{center}
  \caption{\href{https://www.datacamp.com/courses/reporting-with-r-markdown}{Reporting with R Markdown}}\label{fig}
\end{figure}

\section{GitHub using RStudio}\label{github-using-rstudio}

You will learn how to use RStudio to version control your code. All
assignments for this course must be submitted to a GitHub repository.
Click on the link the in the figure caption join the course (Chapter 2
only).

\begin{figure}[H]
  \begin{center}
    \includegraphics[width=1in, height=1in]{../../images/rstudio_ide_2.png} 
  \end{center}
  \caption{\href{https://www.datacamp.com/courses/working-with-the-rstudio-ide-part-2}{Version Control with RStudio IDE (Chapter 2 only)}}\label{fig}
\end{figure}

%\showmatmethods


\bibliography{pinp}
\bibliographystyle{jss}



\end{document}

