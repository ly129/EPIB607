\documentclass[letterpaper,9pt,twoside,printwatermark=false]{pinp}

%% Some pieces required from the pandoc template
\providecommand{\tightlist}{%
  \setlength{\itemsep}{0pt}\setlength{\parskip}{0pt}}

% Use the lineno option to display guide line numbers if required.
% Note that the use of elements such as single-column equations
% may affect the guide line number alignment.

\usepackage[T1]{fontenc}
\usepackage[utf8]{inputenc}

% The geometry package layout settings need to be set here...
\geometry{layoutsize={0.95588\paperwidth,0.98864\paperheight},%
          layouthoffset=0.02206\paperwidth,%
		  layoutvoffset=0.00568\paperheight}

\definecolor{pinpblue}{HTML}{185FAF}  % imagecolorpicker on blue for new R logo
\definecolor{pnasbluetext}{RGB}{101,0,0} %



\title{Assignment 10 - Logistic Regression. Due December 2, 11:59pm 2018}

\author[a]{EPIB607 - Inferential Statistics}

  \affil[a]{Fall 2018, McGill University}

\setcounter{secnumdepth}{5}

% Please give the surname of the lead author for the running footer
\leadauthor{Bhatnagar and Hanley}

% Keywords are not mandatory, but authors are strongly encouraged to provide them. If provided, please include two to five keywords, separated by the pipe symbol, e.g:
 \keywords{  Poisson regression |  Rate ratio |  Rate difference |  Person time |  Offset  }  

\begin{abstract}
In this assignment you will practice logistic regression. Be sure to
state the regression model in terms of parameters first. Use regression
functions to fit these models. Answers should be given in full sentences
(DO NOT just provide the number). All figures should have appropriately
labeled axes, titles and captions (if necessary). Units for means and
CIs should be provided. All graphs and calculations are to be completed
in an R Markdown document. Please submit both the compiled HTML report
and the source file (.Rmd) to myCourses by December 2, 2018, 11:59pm.
Both HTML and .Rmd files should be saved as
`IDnumber\_LastName\_FirstName\_EPIB607\_A10'.
\end{abstract}

\dates{This version was compiled on \today}
\doi{\url{https://sahirbhatnagar.com/EPIB607/}}

\pinpfootercontents{Assignment 10 due December 2, 2018 by 11:59pm}

\begin{document}

% Optional adjustment to line up main text (after abstract) of first page with line numbers, when using both lineno and twocolumn options.
% You should only change this length when you've finalised the article contents.
\verticaladjustment{-2pt}

\maketitle
\thispagestyle{firststyle}
\ifthenelse{\boolean{shortarticle}}{\ifthenelse{\boolean{singlecolumn}}{\abscontentformatted}{\abscontent}}{}

% If your first paragraph (i.e. with the \dropcap) contains a list environment (quote, quotation, theorem, definition, enumerate, itemize...), the line after the list may have some extra indentation. If this is the case, add \parshape=0 to the end of the list environment.


\section*{Template}\label{template}
\addcontentsline{toc}{section}{Template}

There is no template for this assignment. You may use the same template
from previous assignments. Be sure to include your code at the end of
the compiled report.

\section{Kidney Stones}\label{kidney-stones}

The 1986 BMJ article
\textit{Comparison of treatment of renal calculi by open surgery, percutaneous nephrolithotomy, and extracorporeal shockwave lithotripsy}
by Charig et. al, was a study designed to compare different methods of
treating kidney stones in order to establish which was the most cost
effective and successful. The procedure, either open surgery, or
percutaneous nephrolithotomy (PN, a keyhole surgery procedure), was
defined to be successful if stones were eliminated or reduced to less
than 2 mm after three months. The study collected cases of kidney stones
treated at a particular UK hospital during 1972-1985. The counts of
successes for the two surgical procedures, stratified by the size of the
kidney stone (smaller than 2cm/at least 2cm in diameter) were:

\begin{table}[h]
    \centering
    \begin{tabular}{lcc|c}
        $<$ 2cm & Unsuccessful &  Successful & Total\\
        Open surgery & 6 & 81 & 87 \\
        PN           & 36 & 234 & 270 \\
        \hline
        Total   & 42 & 315 & 357 \\
        & & &  \\
                $\geq$ 2cm & Unsuccessful &  Successful & Total\\
        Open surgery & 71 & 192 & 263 \\
        PN           & 25 & 55 & 80 \\
        \hline
        Total       & 96 & 247 & 343
    \end{tabular}
\end{table}

Enter the data into \texttt{R} as frequency records, and complete the
following exercises.

\begin{enumerate}
\def\labelenumi{\alph{enumi}.}
\tightlist
\item
  Fit a logistic regression model for the failure of the surgical
  procedure given the surgery type and adjusting for the size of the
  kidney stone.
\item
  Based on the model output, calculate a 95\% confidence interval for
  the odds ratio of failure of the procedure in open surgery vs.~keyhole
  surgery.
\item
  Based on the model output, calculate the risks of failure and the
  expected numbers of failures in all four patient categories. Assess
  the model fit by comparing the expected failure counts to the observed
  counts using a \(\chi^2\)-goodness of fit test. Note that when
  applying the \(\chi^2\)-test to a logistic regression model, the test
  statistic is of the form \[
  \sum_{i=1}^n \frac{(O_i - E_i)^2}{N_i p_i (1-p_i)} \sim \chi^2_{(n-m)}
  \] where \(n\) is the number of patient categories, \(m\) is the
  number of estimated parameters and \(N_i\) is the total number of
  patients in the category \(i\). The denominator of the test statistic
  had to be adjusted because now the risks \(p_i\) do not sum to one. In
  this context, the goodness of fit test is also sometimes referred to
  as the Hosmer-Lemeshow test.\\
\item
  Fit also a saturated logistic model which includes an interaction term
  between the surgery type and size of the kidney stone. Calculate the
  expected counts based on this model. Why does it not make sense to
  test the goodness of fit of a saturated model?
\item
  Verify that you will get the exact same results as in part (a) by
  transforming the data into individual level records.
\item
  Fit also a log-linear model for the risk of failure, given the type of
  surgery and adjusting for the size of the kidney stone, and based on
  the results, calculate a 95\% confidence interval for the risk ratio
  of failure of the procedure in open surgery vs.~keyhole surgery.
\end{enumerate}

%\showmatmethods


\bibliography{pinp}
\bibliographystyle{jss}



\end{document}

