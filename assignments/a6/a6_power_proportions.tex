\documentclass[letterpaper,9pt,twoside,printwatermark=false]{pinp}

%% Some pieces required from the pandoc template
\providecommand{\tightlist}{%
  \setlength{\itemsep}{0pt}\setlength{\parskip}{0pt}}

% Use the lineno option to display guide line numbers if required.
% Note that the use of elements such as single-column equations
% may affect the guide line number alignment.

\usepackage[T1]{fontenc}
\usepackage[utf8]{inputenc}

% The geometry package layout settings need to be set here...
\geometry{layoutsize={0.95588\paperwidth,0.98864\paperheight},%
          layouthoffset=0.02206\paperwidth,%
		  layoutvoffset=0.00568\paperheight}

\definecolor{pinpblue}{HTML}{185FAF}  % imagecolorpicker on blue for new R logo
\definecolor{pnasbluetext}{RGB}{101,0,0} %



\title{Assignment 6 - Power, Sample Size and Inference for a Population
Proportion. Due October 21, 11:59pm 2018}

\author[a]{EPIB607 - Inferential Statistics}

  \affil[a]{Fall 2018, McGill University}

\setcounter{secnumdepth}{5}

% Please give the surname of the lead author for the running footer
\leadauthor{Bhatnagar and Hanley}

% Keywords are not mandatory, but authors are strongly encouraged to provide them. If provided, please include two to five keywords, separated by the pipe symbol, e.g:
 \keywords{  Power |  Sample Size |  Binomial Distribution |  One sample proportion |  Bootstrap  }  

\begin{abstract}
In this assignment you will practice conducting inference for a one
sample proportion as well as conducting power and sample size
calculations. Answers should be given in full sentences (DO NOT just
provide the number). All figures should have appropriately labeled axes,
titles and captions (if necessary). Units for means and CIs should be
provided. All graphs and calculations are to be completed in an R
Markdown document using the provided template. You are free to choose
any function from any package to complete the assignment. Concise
answers will be rewarded. Be brief and to the point. Please submit both
the compiled HTML report and the source file (.Rmd) to myCourses by
October 21, 2018, 11:59pm. Both HTML and .Rmd files should be saved as
`IDnumber\_LastName\_FirstName\_EPIB607\_A6'.
\end{abstract}

\dates{This version was compiled on \today}
\doi{\url{https://sahirbhatnagar.com/EPIB607/}}

\pinpfootercontents{Assignment 6 due October 21, 2018 by 11:59pm}

\begin{document}

% Optional adjustment to line up main text (after abstract) of first page with line numbers, when using both lineno and twocolumn options.
% You should only change this length when you've finalised the article contents.
\verticaladjustment{-2pt}

\maketitle
\thispagestyle{firststyle}
\ifthenelse{\boolean{shortarticle}}{\ifthenelse{\boolean{singlecolumn}}{\abscontentformatted}{\abscontent}}{}

% If your first paragraph (i.e. with the \dropcap) contains a list environment (quote, quotation, theorem, definition, enumerate, itemize...), the line after the list may have some extra indentation. If this is the case, add \parshape=0 to the end of the list environment.


\section*{Template}\label{template}
\addcontentsline{toc}{section}{Template}

The \texttt{.Rmd} template for Assignment 6 is available
\href{https://github.com/sahirbhatnagar/EPIB607/raw/master/assignments/a6/a6_template.Rmd}{here}

\section{Power and sample size calculations -
1}\label{power-and-sample-size-calculations---1}

Suppose we wished to assess, via a formal statistical test, whether (at
an \textit{population}, rather than an individual, level) a
step-counting device or app is unbiased (\(H_0\)) or under-counts
(\(H_1\)). Suppose we will do so the way Case et al. did, but measuring
\(n\) persons just once each. We observe the device count when the true
count on the treadmill reaches 500. The statistical test will declare
the test `positive' (the departure from 500 is
\textit{statistically significant}) if the mean of the \(n\)
observations is \textit{below} \(500 - 1.96 \times s/n^{1/2}\), where s
is the SD of the \(n\) observations, and the 1.96 (assume the ultimate
\(n\) will be large enough that the \(t\) and \(z\) distributions are
interchangeable) is chosen to give the test a type I error of 5\%. Since
each person is only being measured once, we will not be able to
distinguish the genuine between-person variance, \(\sigma^2_B\) from the
within-person variance, \(\sigma^2_W.\) Thus the sample variance,
\(s^2\) will be an estimate of \(\sigma^2 = \sigma^2_B +\sigma^2_W\).

\begin{enumerate}
\item
Using a planned sample size of $n=25$, and $\sigma = 60$ steps as a pre-study best-guess as to the  $s$ that might  be observed in them, calculate the critical value  $500 - 1.96\times  s/n^{1/2}$.
\item
Now imagine that in an infinite sample,  the mean would not be the null 500, but $\mu=470.$
Calculate the probability that the mean in the sample of 25  will be less than this critical value. (Use the same $s$ for the alternative that you used for the null.)
\item
By trial and error, or better still by deriving a general formula, adjust the $n$ until this probability (i.e., the power) is 80\%. Show the 2 probabilities in a diagram, as was done in Figure 4 in section 4.3.2 of the notes.
\item Show that this $n$ satisfies the equality $$1.96\times SE_{null}[\bar{y}] + 0.84 \times SE_{alt}[\bar{y}] = 30.$$

\end{enumerate}

\newpage

\section{Attitudes toward school}\label{attitudes-toward-school}

The Survey of Study Habits and Attitudes (SSHA) is a psychological test
that measures the motivation, attitude toward school, and study habits
of students. Scores range from 0 to 200. The mean score for U.S. college
students is about 115, and the standard deviation is about 30. A teacher
who suspects that older students have better attitudes toward school
gives the SSHA to 25 students who are at least 30 years of age. Their
mean score is \(\bar{y}\) = 132.2 with a sample standard deviation
\(s = 28\).

\begin{enumerate}
\def\labelenumi{\alph{enumi}.}
\tightlist
\item
  The teacher asks you to carry out a formal statistical test for her
  hypothesis. Perform a test, provide a 95\% confidence interval and
  state your conclusion clearly.
\item
  What assumptions did you use in part (a). Which of these assumptions
  is most important to the validity of your conclusion in part (a).
\end{enumerate}

\section{Does a full moon affect
behavior?}\label{does-a-full-moon-affect-behavior}

Many people believe that the moon influences the actions of some
individuals. A study of dementia patients in nursing homes recorded
various types of disruptive behaviors every day for 12 weeks. Days were
classified as moon days if they were in a 3-day period centered at the
day of the full moon. For each patient, the average number of disruptive
behaviors was computed for moon days and for all other days. The
hypothesis is that moon days will lead to more disruptive behavior. We
look at a data set consisting of observations on 15 dementia patients in
nursing homes (available in the \texttt{fullmoon.csv} file):

\begin{Shaded}
\begin{Highlighting}[]
\NormalTok{fullmoon <-}\StringTok{ }\KeywordTok{read.csv}\NormalTok{(}\StringTok{"fullmoon.csv"}\NormalTok{)}
\end{Highlighting}
\end{Shaded}

\begin{ShadedResult}
\begin{verbatim}
#     patient moon_days other_days
#  1        1      3.33       0.27
#  2        2      3.67       0.59
#  3        3      2.67       0.32
#  4        4      3.33       0.19
#  5        5      3.33       1.26
#  6        6      3.67       0.11
#  7        7      4.67       0.30
#  8        8      2.67       0.40
#  9        9      6.00       1.59
#  10      10      4.33       0.60
#  11      11      3.33       0.65
#  12      12      0.67       0.69
#  13      13      1.33       1.26
#  14      14      0.33       0.23
#  15      15      2.00       0.38
\end{verbatim}
\end{ShadedResult}

\begin{enumerate}
\def\labelenumi{\alph{enumi}.}
\tightlist
\item
  Calculate a 95\% confidence interval for the mean difference in
  disruptive behaviors. State the assumptions you used to calculate this
  interval.
\item
  Calculate a 95\% bootstrap confidence interval for the mean difference
  in disruptive behaviors and compare to the one obtained in part (a).
  Comment on the bootstrap sampling distribution and compare it to the
  assumptions you made in part (a).
\item
  Test the hypothesis that moon days will lead to more disruptive
  behavior. State your assumptions and provide a brief conclusion based
  on your analysis.
\item
  Find the minimum value of the mean difference in disruptive behaviors
  (\(\bar{y}\)) needed to reject the null hypothesis.
\item
  What is the probability of detecting an increase of 1.0 aggressive
  behavior per day during moon days? \emph{Hint: calculate the
  probability of the event calculated in part (d) using a normal
  distribution with \(\mu=1\) and \(\sigma =\) the standard error of the
  mean}
\end{enumerate}

\newpage 

\section{How deep is the ocean?}\label{how-deep-is-the-ocean}

This question is based on the
\href{https://github.com/sahirbhatnagar/EPIB607/blob/master/exercises/water/students/260194225_water_exercise_epib607.pdf}{in-class
Exercise} on sampling distributions and builds on
\href{https://github.com/sahirbhatnagar/EPIB607/raw/master/assignments/a4/a4_clt_ci.pdf}{Question
4 from Assignment 4}. For your sample of \(n=20\) of depths of the ocean

\begin{enumerate}
\def\labelenumi{\alph{enumi}.}
\tightlist
\item
  Calculate a 95\% Confidence interval using the \(t\) procedure
\item
  Plot the qnorm, bootstrap, and \(t\) procedure confidence intervals on
  the same plot and comment on the how the \(t\) interval compares to
  the other 2 intervals. You may use the \texttt{compare\_CI} function
  provided below to produce the plot.
\end{enumerate}

\begin{Shaded}
\begin{Highlighting}[]
\NormalTok{compare_CI <-}\StringTok{ }\ControlFlowTok{function}\NormalTok{(ybar, QNORM, BOOT, TPROCEDURE,}
                       \DataTypeTok{col =} \KeywordTok{c}\NormalTok{(}\StringTok{"#E41A1C"}\NormalTok{,}\StringTok{"#377EB8"}\NormalTok{,}\StringTok{"#4DAF4A"}\NormalTok{)) \{}

\NormalTok{  dt <-}\StringTok{ }\KeywordTok{data.frame}\NormalTok{(}\DataTypeTok{type =} \KeywordTok{c}\NormalTok{(}\StringTok{"qnorm"}\NormalTok{, }\StringTok{"bootstrap"}\NormalTok{, }\StringTok{"t"}\NormalTok{),}
                   \DataTypeTok{ybar =} \KeywordTok{rep}\NormalTok{(ybar, }\DecValTok{3}\NormalTok{),}
                   \DataTypeTok{low =} \KeywordTok{c}\NormalTok{(QNORM[}\DecValTok{1}\NormalTok{], BOOT[}\DecValTok{1}\NormalTok{], TPROCEDURE[}\DecValTok{1}\NormalTok{]),}
                   \DataTypeTok{up =} \KeywordTok{c}\NormalTok{(QNORM[}\DecValTok{2}\NormalTok{], BOOT[}\DecValTok{2}\NormalTok{], TPROCEDURE[}\DecValTok{2}\NormalTok{])}
\NormalTok{  )}
  
  \KeywordTok{plot}\NormalTok{(dt}\OperatorTok{$}\NormalTok{ybar, }\DecValTok{1}\OperatorTok{:}\KeywordTok{nrow}\NormalTok{(dt), }\DataTypeTok{pch =} \DecValTok{20}\NormalTok{, }\DataTypeTok{ylim =} \KeywordTok{c}\NormalTok{(}\DecValTok{0}\NormalTok{, }\DecValTok{5}\NormalTok{), }
       \DataTypeTok{xlim =} \KeywordTok{range}\NormalTok{(}\KeywordTok{pretty}\NormalTok{(}\KeywordTok{c}\NormalTok{(dt}\OperatorTok{$}\NormalTok{low, dt}\OperatorTok{$}\NormalTok{up))),}
       \DataTypeTok{xlab =} \StringTok{"Depth of ocean (m)"}\NormalTok{, }\DataTypeTok{ylab =} \StringTok{"Confidence Interval Type"}\NormalTok{,}
       \DataTypeTok{las =} \DecValTok{1}\NormalTok{, }\DataTypeTok{cex.axis =} \FloatTok{0.8}\NormalTok{, }\DataTypeTok{cex =} \DecValTok{3}\NormalTok{)}
  
  \KeywordTok{abline}\NormalTok{(}\DataTypeTok{v =} \DecValTok{37}\NormalTok{, }\DataTypeTok{lty =} \DecValTok{2}\NormalTok{, }\DataTypeTok{col =} \StringTok{"black"}\NormalTok{, }\DataTypeTok{lwd =} \DecValTok{2}\NormalTok{)}
  \KeywordTok{segments}\NormalTok{(}\DataTypeTok{x0 =}\NormalTok{ dt}\OperatorTok{$}\NormalTok{low, }\DataTypeTok{x1 =}\NormalTok{ dt}\OperatorTok{$}\NormalTok{up,}
           \DataTypeTok{y0 =} \DecValTok{1}\OperatorTok{:}\KeywordTok{nrow}\NormalTok{(dt), }\DataTypeTok{lend =} \DecValTok{1}\NormalTok{,}
           \DataTypeTok{col =}\NormalTok{ col, }\DataTypeTok{lwd =} \DecValTok{4}\NormalTok{)}
  
  \KeywordTok{legend}\NormalTok{(}\StringTok{"topleft"}\NormalTok{,}
         \DataTypeTok{legend =} \KeywordTok{c}\NormalTok{(}\KeywordTok{eval}\NormalTok{(}\KeywordTok{substitute}\NormalTok{( }\KeywordTok{expression}\NormalTok{(}\KeywordTok{paste}\NormalTok{(mu,}\StringTok{" = "}\NormalTok{,}\DecValTok{37}\NormalTok{)))),}
                    \KeywordTok{sprintf}\NormalTok{(}\StringTok{"qnorm CI: [%.f, %.f]"}\NormalTok{,QNORM[}\DecValTok{1}\NormalTok{], QNORM[}\DecValTok{2}\NormalTok{]),}
                    \KeywordTok{sprintf}\NormalTok{(}\StringTok{"bootstrap CI: [%.f, %.f]"}\NormalTok{,BOOT[}\DecValTok{1}\NormalTok{], BOOT[}\DecValTok{2}\NormalTok{]),}
                    \KeywordTok{sprintf}\NormalTok{(}\StringTok{"t CI: [%.f, %.f]"}\NormalTok{,TPROCEDURE[}\DecValTok{1}\NormalTok{], TPROCEDURE[}\DecValTok{2}\NormalTok{])),}
         \DataTypeTok{lty =} \KeywordTok{c}\NormalTok{(}\DecValTok{1}\NormalTok{,}\DecValTok{1}\NormalTok{,}\DecValTok{1}\NormalTok{,}\DecValTok{1}\NormalTok{),}
         \DataTypeTok{col =} \KeywordTok{c}\NormalTok{(}\StringTok{"black"}\NormalTok{,col), }\DataTypeTok{lwd =} \DecValTok{4}\NormalTok{)}
\NormalTok{\}}

\CommentTok{# example of how to use the function:}
\KeywordTok{compare_CI}\NormalTok{(}\DataTypeTok{ybar =} \DecValTok{36}\NormalTok{, }\DataTypeTok{QNORM =} \KeywordTok{c}\NormalTok{(}\DecValTok{25}\NormalTok{,}\DecValTok{40}\NormalTok{), }\DataTypeTok{BOOT =} \KeywordTok{c}\NormalTok{(}\DecValTok{31}\NormalTok{, }\DecValTok{38}\NormalTok{), }\DataTypeTok{TPROCEDURE =} \KeywordTok{c}\NormalTok{(}\DecValTok{28}\NormalTok{, }\DecValTok{40}\NormalTok{))}
\end{Highlighting}
\end{Shaded}

%\showmatmethods


\bibliography{pinp}
\bibliographystyle{jss}



\end{document}

