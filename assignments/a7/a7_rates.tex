\documentclass[letterpaper,9pt,twoside,printwatermark=false]{pinp}

%% Some pieces required from the pandoc template
\providecommand{\tightlist}{%
  \setlength{\itemsep}{0pt}\setlength{\parskip}{0pt}}

% Use the lineno option to display guide line numbers if required.
% Note that the use of elements such as single-column equations
% may affect the guide line number alignment.

\usepackage[T1]{fontenc}
\usepackage[utf8]{inputenc}

% The geometry package layout settings need to be set here...
\geometry{layoutsize={0.95588\paperwidth,0.98864\paperheight},%
          layouthoffset=0.02206\paperwidth,%
		  layoutvoffset=0.00568\paperheight}

\definecolor{pinpblue}{HTML}{185FAF}  % imagecolorpicker on blue for new R logo
\definecolor{pnasbluetext}{RGB}{101,0,0} %



\title{Assignment 7 - Inference about a Population Rate (\(\lambda\)). Due
November 9, 11:59pm 2018}

\author[a]{EPIB607 - Inferential Statistics}

  \affil[a]{Fall 2018, McGill University}

\setcounter{secnumdepth}{5}

% Please give the surname of the lead author for the running footer
\leadauthor{Bhatnagar and Hanley}

% Keywords are not mandatory, but authors are strongly encouraged to provide them. If provided, please include two to five keywords, separated by the pipe symbol, e.g:
 \keywords{  Poisson distribution |  Rates |  Intensity |  Bootstrap  }  

\begin{abstract}
In this assignment you will practice conducting inference for a one
sample rate. Answers should be given in full sentences (DO NOT just
provide the number). All figures should have appropriately labeled axes,
titles and captions (if necessary). Units for means and CIs should be
provided. State your hypotheses and assumptions when applicable. All
graphs and calculations are to be completed in an R Markdown document
using the provided template. You are free to choose any function from
any package to complete the assignment. Concise answers will be
rewarded. Be brief and to the point. Please submit both the compiled
HTML report and the source file (.Rmd) to myCourses by November 9, 2018,
11:59pm. Both HTML and .Rmd files should be saved as
`IDnumber\_LastName\_FirstName\_EPIB607\_A7'.
\end{abstract}

\dates{This version was compiled on \today}
\doi{\url{https://sahirbhatnagar.com/EPIB607/}}

\pinpfootercontents{Assignment 7 due November 9, 2018 by 11:59pm}

\begin{document}

% Optional adjustment to line up main text (after abstract) of first page with line numbers, when using both lineno and twocolumn options.
% You should only change this length when you've finalised the article contents.
\verticaladjustment{-2pt}

\maketitle
\thispagestyle{firststyle}
\ifthenelse{\boolean{shortarticle}}{\ifthenelse{\boolean{singlecolumn}}{\abscontentformatted}{\abscontent}}{}

% If your first paragraph (i.e. with the \dropcap) contains a list environment (quote, quotation, theorem, definition, enumerate, itemize...), the line after the list may have some extra indentation. If this is the case, add \parshape=0 to the end of the list environment.


\section*{Template}\label{template}
\addcontentsline{toc}{section}{Template}

The \texttt{.Rmd} template for Assignment 7 is available
\href{https://github.com/sahirbhatnagar/EPIB607/raw/master/assignments/a7/a7_template.Rmd}{here}

\section{Cancers near Pickering nuclear
station}\label{cancers-near-pickering-nuclear-station}

The observed number of cases of leukemia was 18 in the Pickering nuclear
station experience. We want to test if this observed number is
compatible with the expected number of 12.8 cases, or whether it is
greater than expected.

\begin{enumerate}
\def\labelenumi{\alph{enumi}.}
\tightlist
\item
  Calculate the standardized incidence ratio (SIR).
\item
  State the null and alternative hypotheses. Calculate the p-value for
  this test and provide a statement about the evidence.
\item
  Compute a 95\% CI for the SIR. Does your statement in part (b) agree
  with the 95\% CI?
\end{enumerate}

\section{Self-reported percutaneous injuries in
interns}\label{self-reported-percutaneous-injuries-in-interns}

Read
\href{https://www.dropbox.com/s/b5q7vqo2ev6k2me/EPIB607intensity-model-inference-plan-2018.pdf?dl=0}{Section
6.3 of JH notes on rates}. Reproduce the CIs for the 3 P's (Pediatrics
Psychiatry Pathology) in Table 1. Show your calculations. You may not
use canned functions.

\section{RCT of HPV vaccine}\label{rct-of-hpv-vaccine}

This question is based on Table 3 of the article
\href{https://www.nejm.org/doi/pdf/10.1056/NEJMoa020586}{A Controlled
trail of a Human Papillomavirus Type 16 Vaccine. Laura A. Koutsky et
al., for The Proof of Principle Study Investigators. The New England
Journal of Medicine Vol 347 Nov 21, 2002, p1645} (also available on
myCourses).

\begin{enumerate}

\item[a.]

For the primary per-protocol efficacy analysis, calculate a 95\% CI to accompany the point estimates of infection rate (per 100 woman years) for both the HPV-16 vaccine and placebo groups. Show your calculations. You may not use canned functions. 


\item[b.] 

Based on your results in part (a), is there evidence to suggest that the HPV-16 vaccine reduced infection rates? Explain. 










\end{enumerate}

\newpage

\section{Deaths by Horsekicks in the Prussian
Army}\label{deaths-by-horsekicks-in-the-prussian-army}

This question is in reference to the Deaths by Horsekicks in the
Prussian Army example presented in the
\href{https://github.com/sahirbhatnagar/EPIB607/raw/master/slides/one_sample_rate/EPIB607_one_sample_rate.pdf}{slides
on one sample rates}. The following dataset gives the number of deaths
by horsekicks, by year and army corps:

\begin{Shaded}
\begin{Highlighting}[]
\NormalTok{horsekicks <-}\StringTok{ }\KeywordTok{read.csv}\NormalTok{(}\StringTok{"horsekicks.csv"}\NormalTok{)}
\KeywordTok{head}\NormalTok{(horsekicks)}
\end{Highlighting}
\end{Shaded}

\begin{ShadedResult}
\begin{verbatim}
#    year corps1 corps2 corps3 corps4 corps5 corps6 corps7 corps8 corps9
#  1 1875      0      0      0      0      0      0      0      1      1
#  2 1876      2      0      0      0      1      0      0      0      0
#  3 1877      2      0      0      0      0      0      1      1      0
#  4 1878      1      2      2      1      1      0      0      0      0
#  5 1879      0      0      0      1      1      2      2      0      1
#  6 1880      0      3      2      1      1      1      0      0      0
#    corps10 corps111 corps12 corps13 corps14 total
#  1       0        0       0       1       0     3
#  2       0        0       0       1       1     5
#  3       0        1       0       2       0     7
#  4       0        1       0       1       0     9
#  5       0        0       2       1       0    10
#  6       2        1       4       3       0    18
\end{verbatim}
\end{ShadedResult}

\begin{enumerate}

\item[a.] Create a table of observed and expected frequencies, like the one shown in the \href{https://github.com/sahirbhatnagar/EPIB607/raw/master/slides/one_sample_rate/EPIB607_one_sample_rate.pdf}{slides on one sample rates}

\item[b.] Calculate a 95\% CI for the rate parameter.

\item[c.] (\textbf{BONUS}) Compute a goodness of fit statistic that measures the discrepancy between the observed and expected frequencies. Be sure to group the categories that have low expected values. Include in your answer, the sources you used to help you with this calculation. Write a sentence about your findings.
\end{enumerate}

\section{Crimes on full moon}\label{crimes-on-full-moon}

This question is based on Tables I and II of the article:
\href{https://doi.org/10.1136/bmj.289.6460.1789}{Full moon and crime.Br
Med J (Clin Res Ed) 1984; 289} (also available on myCourses). The
incidence of crimes reported to three police stations in different towns
(one rural, one urban, one industrial) was studied to see if it varied
with the day of the lunar cycle. The period of the study covered
1978-1982.

\begin{enumerate}

\item[a.] Compare crime rates on FullMoon vs. NotFullMoon days (Table I) statistically by calculating a confidence interval for the daily rates in the two segments of time and seeing if they overlap. 

\item[b.] We don't recommend this way of testing differences; instead we recommend computing a single
CI for the rate difference or rate ratio. Following the example of the comparison of bone mineral losses (A4, Q3), compute a CI for the ratio and difference of daily rates by bootstrapping.


\end{enumerate}

%\showmatmethods


\bibliography{pinp}
\bibliographystyle{jss}



\end{document}

