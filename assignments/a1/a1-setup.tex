\documentclass[letterpaper,9pt,twocolumn,twoside,printwatermark=false]{pinp}

%% Some pieces required from the pandoc template
\providecommand{\tightlist}{%
  \setlength{\itemsep}{0pt}\setlength{\parskip}{0pt}}

% Use the lineno option to display guide line numbers if required.
% Note that the use of elements such as single-column equations
% may affect the guide line number alignment.

\usepackage[T1]{fontenc}
\usepackage[utf8]{inputenc}

% The geometry package layout settings need to be set here...
\geometry{layoutsize={0.95588\paperwidth,0.98864\paperheight},%
          layouthoffset=0.02206\paperwidth,%
		  layoutvoffset=0.00568\paperheight}

\definecolor{pinpblue}{HTML}{185FAF}  % imagecolorpicker on blue for new R logo
\definecolor{pnasbluetext}{RGB}{101,0,0} %



\title{Assignment 1 - Setting up the computing environment. Due September 20,
2018.}

\author[a]{EPIB607 - Inferential Statistics}

  \affil[a]{Fall 2018, McGill University}

\setcounter{secnumdepth}{5}

% Please give the surname of the lead author for the running footer
\leadauthor{Author and Author}

% Keywords are not mandatory, but authors are strongly encouraged to provide them. If provided, please include two to five keywords, separated by the pipe symbol, e.g:
 \keywords{  R |  RStudio |  Git |  GitHub |  R Markdown |  DataCamp  }  

\begin{abstract}
Computing is an essential part of modern statistics. However, before
doing any data analysis, we must first install the necessary tools. In
this assignment you will first signup for a free DataCamp account and
login to your McGill GitHub account. Then you will complete a series of
DataCamp courses that will guide you through installing R, RStudio and
Git. After installation, you will learn how to use the RStudio IDE,
create projects and run some basic commands. You will then be introduced
to R Markdown which is a tool for creating reproducible reports. All
future assignments for this course must be in R Markdown format and
submitted to a private online GitHub repository.
\end{abstract}

\dates{This version was compiled on \today}
\doi{\url{https://sahirbhatnagar.com/EPIB607/}}

\pinpfootercontents{Assignment 1 due Sepetember 20, 2018 by 5pm}

\begin{document}

% Optional adjustment to line up main text (after abstract) of first page with line numbers, when using both lineno and twocolumn options.
% You should only change this length when you've finalised the article contents.
\verticaladjustment{-2pt}

\maketitle
\thispagestyle{firststyle}
\ifthenelse{\boolean{shortarticle}}{\ifthenelse{\boolean{singlecolumn}}{\abscontentformatted}{\abscontent}}{}

% If your first paragraph (i.e. with the \dropcap) contains a list environment (quote, quotation, theorem, definition, enumerate, itemize...), the line after the list may have some extra indentation. If this is the case, add \parshape=0 to the end of the list environment.


\section*{Marking}\label{marking}
\addcontentsline{toc}{section}{Marking}

Your progress and completion of these courses will be availble to us
automatically through the DataCamp website. \textbf{You do not need to
hand anything in for this Assignment}. You will receive full credits for
this assignment once we have seen that all tasks have been completed.

\section{Sign up for DataCamp and
GitHub}\label{sign-up-for-datacamp-and-github}

\textbf{Please try to complete this step before class on Thursday
September 6}:

\begin{enumerate}
\def\labelenumi{\arabic{enumi}.}
\tightlist
\item
  Sign up for a free DataCamp account at
  \href{https://www.datacamp.com/groups/shared_links/4c7d78a632b557dfdd6618b3e8fac09495571fec}{this
  link}. Note: you are required to sign up with a
  \texttt{@mail.mcgill.ca} or \texttt{@mcgill.ca} email address.\\
\item
  Sign in to \url{https://github.mcgill.ca/} using your McGill email
  address and corresponding password.
\end{enumerate}

\section{Install Git}\label{install-git}

You need to first install the \href{https://git-scm.com/}{git} version
control system on your system. Follow Chapter 1: Installing Git
\href{https://plot.ly/r/github-getting-started-for-data-scientists/\#chapter-1-installing-git}{at
this link} for step-by-step installation instructions with screenshots.

\section{Install R and RStudio}\label{install-r-and-rstudio}

This short course will guide you through installing both
\href{https://cran.r-project.org/}{R} and
\href{https://www.rstudio.com/products/rstudio/download/preview/}{RStudio}.
RStudio is a software application that facilitates how you interact with
\texttt{R}. Click on the link the in the figure caption join the course.

\begin{figure}[H]
  \begin{center}
    \includegraphics[width=1in, height=1in]{../../images/rstudio_ide.png} 
  \end{center}
  \caption{\href{https://www.datacamp.com/courses/working-with-the-rstudio-ide-part-1}{Working with the RStudio IDE (Part 1)}}\label{fig}
\end{figure}

\section{Introduction to R}\label{introduction-to-r}

In this course you will get a hands-on introduction to the basic
commands in R. With the knowledge gained in this course, you will be
ready to perform a data analysis. Click on the link the in the figure
caption join the course.

\begin{figure}[H]
  \begin{center}
    \includegraphics[width=1in, height=1in]{../../images/intro_r.png} 
  \end{center}
  \caption{\href{https://www.datacamp.com/courses/free-introduction-to-r}{Introduction to R}}\label{fig}
\end{figure}

\section{R Markdown}\label{r-markdown}

You will learn how to create reproducible reports using R and Markdown.
All assignments for this course must be submitted in this format. Click
on the link the in the figure caption join the course.

\begin{figure}[H]
  \begin{center}
    \includegraphics[width=1in, height=1in]{../../images/rmarkdown_r.png} 
  \end{center}
  \caption{\href{https://www.datacamp.com/courses/reporting-with-r-markdown}{Reporting with R Markdown}}\label{fig}
\end{figure}

\section{GitHub using RStudio}\label{github-using-rstudio}

You will learn how to use RStudio to version control your code. All
assignments for this course must be submitted to a GitHub repository.
Click on the link the in the figure caption join the course (Chapter 2
only).

\begin{figure}[H]
  \begin{center}
    \includegraphics[width=1in, height=1in]{../../images/rstudio_ide_2.png} 
  \end{center}
  \caption{\href{https://www.datacamp.com/courses/working-with-the-rstudio-ide-part-2}{Version Control with RStudio IDE (Chapter 2 only)}}\label{fig}
\end{figure}

%\showmatmethods


\bibliography{pinp}
\bibliographystyle{jss}



\end{document}

