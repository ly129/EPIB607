\documentclass[letterpaper,9pt,twoside,printwatermark=false]{pinp}

%% Some pieces required from the pandoc template
\providecommand{\tightlist}{%
  \setlength{\itemsep}{0pt}\setlength{\parskip}{0pt}}

% Use the lineno option to display guide line numbers if required.
% Note that the use of elements such as single-column equations
% may affect the guide line number alignment.

\usepackage[T1]{fontenc}
\usepackage[utf8]{inputenc}

% The geometry package layout settings need to be set here...
\geometry{layoutsize={0.95588\paperwidth,0.98864\paperheight},%
          layouthoffset=0.02206\paperwidth,%
		  layoutvoffset=0.00568\paperheight}

\definecolor{pinpblue}{HTML}{185FAF}  % imagecolorpicker on blue for new R logo
\definecolor{pnasbluetext}{RGB}{101,0,0} %



\title{Assignment 5 - Inference for a Population Mean. Due October 14, 11:59pm
2018}

\author[a]{EPIB607 - Inferential Statistics}

  \affil[a]{Fall 2018, McGill University}

\setcounter{secnumdepth}{5}

% Please give the surname of the lead author for the running footer
\leadauthor{Bhatnagar and Hanley}

% Keywords are not mandatory, but authors are strongly encouraged to provide them. If provided, please include two to five keywords, separated by the pipe symbol, e.g:
 \keywords{  t-test |  One sample mean |  Bootstrap  }  

\begin{abstract}
In this assignment you will practice conducting inference for a one
sample mean using either the z procedure, t procedure, or the bootstrap.
Answers should be given in full sentences (DO NOT just provide the
number). All figures should have appropriately labeled axes, titles and
captions (if necessary). All graphs and calculations are to be completed
in an R Markdown document using the provided template. You are free to
choose any function from any package to complete the assignment. Concise
answers will be rewarded. Be brief and to the point. Please submit both
the compiled HTML report and the source file (.Rmd) to myCourses by
October 14, 2018, 11:59pm. Both HTML and .Rmd files should be saved as
`IDnumber\_LastName\_FirstName\_EPIB607\_A5'.
\end{abstract}

\dates{This version was compiled on \today}
\doi{\url{https://sahirbhatnagar.com/EPIB607/}}

\pinpfootercontents{Assignment 5 due October 14, 2018 by 11:59pm}

\begin{document}

% Optional adjustment to line up main text (after abstract) of first page with line numbers, when using both lineno and twocolumn options.
% You should only change this length when you've finalised the article contents.
\verticaladjustment{-2pt}

\maketitle
\thispagestyle{firststyle}
\ifthenelse{\boolean{shortarticle}}{\ifthenelse{\boolean{singlecolumn}}{\abscontentformatted}{\abscontent}}{}

% If your first paragraph (i.e. with the \dropcap) contains a list environment (quote, quotation, theorem, definition, enumerate, itemize...), the line after the list may have some extra indentation. If this is the case, add \parshape=0 to the end of the list environment.


\section*{Template}\label{template}
\addcontentsline{toc}{section}{Template}

The \texttt{.Rmd} template for Assignment 5 is available
\href{https://github.com/sahirbhatnagar/EPIB607/raw/master/assignments/a5/a5_template.Rmd}{here}

\section{Food intake and weight gain}\label{food-intake-and-weight-gain}

If we increase our food intake, we generally gain weight. Nutrition
scientists can calculate the amount of weight gain that would be
associated with a given increase in calories. In one study, 16 nonobese
adults, aged 25 to 36 years, were fed 1000 calories per day in excess of
the calories needed to maintain a stable body weight. The subjects
maintained this diet for 8 weeks, so they consumed a total of 56,000
extra calories. According to theory, 3500 extra calories will translate
into a weight gain of 1 pound. Therfore we expect each of these subjects
to gain 56,000/3500=16 pounds (lb). Here are the weights (given in the
\texttt{weightgain.csv} file) before and after the 8-week period
expressed in kilograms (kg):

\begin{Shaded}
\begin{Highlighting}[]
\NormalTok{weight <-}\StringTok{ }\KeywordTok{read.csv}\NormalTok{(}\StringTok{"weightgain.csv"}\NormalTok{)}
\end{Highlighting}
\end{Shaded}

\begin{ShadedResult}
\begin{verbatim}
#     subject before after
#  1        1   55.7  61.7
#  2        2   54.9  58.8
#  3        3   59.6  66.0
#  4        4   62.3  66.2
#  5        5   74.2  79.0
#  6        6   75.6  82.3
#  7        7   70.7  74.3
#  8        8   53.3  59.3
#  9        9   73.3  79.1
#  10      10   63.4  66.0
#  11      11   68.1  73.4
#  12      12   73.7  76.9
#  13      13   91.7  93.1
#  14      14   55.9  63.0
#  15      15   61.7  68.2
#  16      16   57.8  60.3
\end{verbatim}
\end{ShadedResult}

\begin{enumerate}
\def\labelenumi{\alph{enumi}.}
\tightlist
\item
  Calculate a 95\% confidence interval for the mean weight change and
  give a sentence explaining the meaning of the 95\%. State your
  assumptions.
\item
  Calculate a 95\% bootstrap confidence interval for the mean weight
  change and compare it to the one obtained in part (a). Comment on the
  bootstrap sampling distribution and compare it to the assumptions you
  made in part (a).
\item
  Convert the units of the mean weight gain and 95\% confidence interval
  to pounds. Note that 1 kilogram is equal to 2.2 pounds.
\item
  Test the null hypothesis that the mean weight gain is 16 lbs. State
  your assumptions and justify your choice of test. Be sure to specify
  the null and alternative hypotheses. What do you conclude?
\end{enumerate}

\newpage

\section{Attitudes toward school}\label{attitudes-toward-school}

The Survey of Study Habits and Attitudes (SSHA) is a psychological test
that measures the motivation, attitude toward school, and study habits
of students. Scores range from 0 to 200. The mean score for U.S. college
students is about 115, and the standard deviation is about 30. A teacher
who suspects that older students have better attitudes toward school
gives the SSHA to 25 students who are at least 30 years of age. Their
mean score is \(\bar{y}\) = 132.2 with a sample standard deviation
\(s = 28\).

\begin{enumerate}
\def\labelenumi{\alph{enumi}.}
\tightlist
\item
  The teacher asks you to carry out a formal statistical test for her
  hypothesis. Perform a test, provide a 95\% confidence interval and
  state your conclusion clearly.
\item
  What assumptions did you use in part (a). Which of these assumptions
  is most important to the validity of your conclusion in part (a).
\end{enumerate}

\section{Does a full moon affect
behavior?}\label{does-a-full-moon-affect-behavior}

Many people believe that the moon influences the actions of some
individuals. A study of dementia patients in nursing homes recorded
various types of disruptive behaviors every day for 12 weeks. Days were
classified as moon days if they were in a 3-day period centered at the
day of the full moon. For each patient, the average number of disruptive
behaviors was computed for moon days and for all other days. The
hypothesis is that moon days will lead to more disruptive behavior. We
look at a data set consisting of observations on 15 dementia patients in
nursing homes (available in the \texttt{fullmoon.csv} file):

\begin{Shaded}
\begin{Highlighting}[]
\NormalTok{fullmoon <-}\StringTok{ }\KeywordTok{read.csv}\NormalTok{(}\StringTok{"fullmoon.csv"}\NormalTok{)}
\end{Highlighting}
\end{Shaded}

\begin{ShadedResult}
\begin{verbatim}
#     patient moon_days other_days
#  1        1      3.33       0.27
#  2        2      3.67       0.59
#  3        3      2.67       0.32
#  4        4      3.33       0.19
#  5        5      3.33       1.26
#  6        6      3.67       0.11
#  7        7      4.67       0.30
#  8        8      2.67       0.40
#  9        9      6.00       1.59
#  10      10      4.33       0.60
#  11      11      3.33       0.65
#  12      12      0.67       0.69
#  13      13      1.33       1.26
#  14      14      0.33       0.23
#  15      15      2.00       0.38
\end{verbatim}
\end{ShadedResult}

\begin{enumerate}
\def\labelenumi{\alph{enumi}.}
\tightlist
\item
  Calculate a 95\% confidence interval for the mean difference in
  disruptive behaviors. State the assumptions you used to calculate this
  interval.
\item
  Calculate a 95\% bootstrap confidence interval for the mean difference
  in disruptive behaviors and compare to the one obtained in part (a).
  Comment on the bootstrap sampling distribution and compare it to the
  assumptions you made in part (a).
\item
  Test the hypothesis that moon days will lead to more disruptive
  behavior. State your assumptions and provide a brief conclusion based
  on your analysis.
\item
  Find the minimum value of the mean difference in disruptive behaviors
  (\(\bar{y}\)) needed to reject the null hypothesis.
\item
  What is the probability of detecting an increase of 1.0 aggressive
  behavior per day during moon days? \emph{Hint: calculate the
  probability of the event calculated in part (d) using a normal
  distribution with \(\mu=1\) and \(\sigma =\) the standard error of the
  mean}
\end{enumerate}

\newpage 

\section{How deep is the ocean?}\label{how-deep-is-the-ocean}

This question is based on the
\href{https://github.com/sahirbhatnagar/EPIB607/blob/master/exercises/water/students/260194225_water_exercise_epib607.pdf}{in-class
Exercise} on sampling distributions and builds on
\href{https://github.com/sahirbhatnagar/EPIB607/raw/master/assignments/a4/a4_clt_ci.pdf}{Question
4 from Assignment 4}. For your sample of \(n=20\) of depths of the ocean

\begin{enumerate}
\def\labelenumi{\alph{enumi}.}
\tightlist
\item
  Calculate a 95\% Confidence interval using the \(t\) procedure
\item
  Plot the qnorm, bootstrap, and \(t\) procedure confidence intervals on
  the same plot and comment on the how the \(t\) interval compares to
  the other 2 intervals. You may use the \texttt{compare\_CI} function
  provided below to produce the plot.
\end{enumerate}

\begin{Shaded}
\begin{Highlighting}[]
\NormalTok{compare_CI <-}\StringTok{ }\ControlFlowTok{function}\NormalTok{(ybar, QNORM, BOOT, TPROCEDURE,}
                       \DataTypeTok{col =} \KeywordTok{c}\NormalTok{(}\StringTok{"#E41A1C"}\NormalTok{,}\StringTok{"#377EB8"}\NormalTok{,}\StringTok{"#4DAF4A"}\NormalTok{)) \{}

\NormalTok{  dt <-}\StringTok{ }\KeywordTok{data.frame}\NormalTok{(}\DataTypeTok{type =} \KeywordTok{c}\NormalTok{(}\StringTok{"qnorm"}\NormalTok{, }\StringTok{"bootstrap"}\NormalTok{, }\StringTok{"t"}\NormalTok{),}
                   \DataTypeTok{ybar =} \KeywordTok{rep}\NormalTok{(ybar, }\DecValTok{3}\NormalTok{),}
                   \DataTypeTok{low =} \KeywordTok{c}\NormalTok{(QNORM[}\DecValTok{1}\NormalTok{], BOOT[}\DecValTok{1}\NormalTok{], TPROCEDURE[}\DecValTok{1}\NormalTok{]),}
                   \DataTypeTok{up =} \KeywordTok{c}\NormalTok{(QNORM[}\DecValTok{2}\NormalTok{], BOOT[}\DecValTok{2}\NormalTok{], TPROCEDURE[}\DecValTok{2}\NormalTok{])}
\NormalTok{  )}
  
  \KeywordTok{plot}\NormalTok{(dt}\OperatorTok{$}\NormalTok{ybar, }\DecValTok{1}\OperatorTok{:}\KeywordTok{nrow}\NormalTok{(dt), }\DataTypeTok{pch =} \DecValTok{20}\NormalTok{, }\DataTypeTok{ylim =} \KeywordTok{c}\NormalTok{(}\DecValTok{0}\NormalTok{, }\DecValTok{5}\NormalTok{), }
       \DataTypeTok{xlim =} \KeywordTok{range}\NormalTok{(}\KeywordTok{pretty}\NormalTok{(}\KeywordTok{c}\NormalTok{(dt}\OperatorTok{$}\NormalTok{low, dt}\OperatorTok{$}\NormalTok{up))),}
       \DataTypeTok{xlab =} \StringTok{"Depth of ocean (m)"}\NormalTok{, }\DataTypeTok{ylab =} \StringTok{"Confidence Interval Type"}\NormalTok{,}
       \DataTypeTok{las =} \DecValTok{1}\NormalTok{, }\DataTypeTok{cex.axis =} \FloatTok{0.8}\NormalTok{, }\DataTypeTok{cex =} \DecValTok{3}\NormalTok{)}
  
  \KeywordTok{abline}\NormalTok{(}\DataTypeTok{v =} \DecValTok{37}\NormalTok{, }\DataTypeTok{lty =} \DecValTok{2}\NormalTok{, }\DataTypeTok{col =} \StringTok{"black"}\NormalTok{, }\DataTypeTok{lwd =} \DecValTok{2}\NormalTok{)}
  \KeywordTok{segments}\NormalTok{(}\DataTypeTok{x0 =}\NormalTok{ dt}\OperatorTok{$}\NormalTok{low, }\DataTypeTok{x1 =}\NormalTok{ dt}\OperatorTok{$}\NormalTok{up,}
           \DataTypeTok{y0 =} \DecValTok{1}\OperatorTok{:}\KeywordTok{nrow}\NormalTok{(dt), }\DataTypeTok{lend =} \DecValTok{1}\NormalTok{,}
           \DataTypeTok{col =}\NormalTok{ col, }\DataTypeTok{lwd =} \DecValTok{4}\NormalTok{)}
  
  \KeywordTok{legend}\NormalTok{(}\StringTok{"topleft"}\NormalTok{,}
         \DataTypeTok{legend =} \KeywordTok{c}\NormalTok{(}\KeywordTok{eval}\NormalTok{(}\KeywordTok{substitute}\NormalTok{( }\KeywordTok{expression}\NormalTok{(}\KeywordTok{paste}\NormalTok{(mu,}\StringTok{" = "}\NormalTok{,}\DecValTok{37}\NormalTok{)))),}
                    \KeywordTok{sprintf}\NormalTok{(}\StringTok{"qnorm CI: [%.f, %.f]"}\NormalTok{,QNORM[}\DecValTok{1}\NormalTok{], QNORM[}\DecValTok{2}\NormalTok{]),}
                    \KeywordTok{sprintf}\NormalTok{(}\StringTok{"bootstrap CI: [%.f, %.f]"}\NormalTok{,BOOT[}\DecValTok{1}\NormalTok{], BOOT[}\DecValTok{2}\NormalTok{]),}
                    \KeywordTok{sprintf}\NormalTok{(}\StringTok{"t CI: [%.f, %.f]"}\NormalTok{,TPROCEDURE[}\DecValTok{1}\NormalTok{], TPROCEDURE[}\DecValTok{2}\NormalTok{])),}
         \DataTypeTok{lty =} \KeywordTok{c}\NormalTok{(}\DecValTok{1}\NormalTok{,}\DecValTok{1}\NormalTok{,}\DecValTok{1}\NormalTok{,}\DecValTok{1}\NormalTok{),}
         \DataTypeTok{col =} \KeywordTok{c}\NormalTok{(}\StringTok{"black"}\NormalTok{,col), }\DataTypeTok{lwd =} \DecValTok{4}\NormalTok{)}
\NormalTok{\}}

\CommentTok{# example of how to use the function:}
\KeywordTok{compare_CI}\NormalTok{(}\DataTypeTok{ybar =} \DecValTok{36}\NormalTok{, }\DataTypeTok{QNORM =} \KeywordTok{c}\NormalTok{(}\DecValTok{25}\NormalTok{,}\DecValTok{40}\NormalTok{), }\DataTypeTok{BOOT =} \KeywordTok{c}\NormalTok{(}\DecValTok{31}\NormalTok{, }\DecValTok{38}\NormalTok{), }\DataTypeTok{TPROCEDURE =} \KeywordTok{c}\NormalTok{(}\DecValTok{28}\NormalTok{, }\DecValTok{40}\NormalTok{))}
\end{Highlighting}
\end{Shaded}

%\showmatmethods


\bibliography{pinp}
\bibliographystyle{jss}



\end{document}

