\documentclass[letterpaper,9pt,twoside,printwatermark=false]{pinp}

%% Some pieces required from the pandoc template
\providecommand{\tightlist}{%
  \setlength{\itemsep}{0pt}\setlength{\parskip}{0pt}}

% Use the lineno option to display guide line numbers if required.
% Note that the use of elements such as single-column equations
% may affect the guide line number alignment.

\usepackage[T1]{fontenc}
\usepackage[utf8]{inputenc}
\usepackage{longtable}

% The geometry package layout settings need to be set here...
\geometry{layoutsize={0.95588\paperwidth,0.98864\paperheight},%
          layouthoffset=0.02206\paperwidth,%
		  layoutvoffset=0.00568\paperheight}

\definecolor{pinpblue}{HTML}{185FAF}  % imagecolorpicker on blue for new R logo
\definecolor{pnasbluetext}{RGB}{101,0,0} %



\title{Assignment 3 - Sampling Distributions. Due September 28, 11:59pm 2018}

\author[a]{EPIB607 - Inferential Statistics}

  \affil[a]{Fall 2018, McGill University}

\setcounter{secnumdepth}{5}

% Please give the surname of the lead author for the running footer
\leadauthor{Bhatnagar and Hanley}

% Keywords are not mandatory, but authors are strongly encouraged to provide them. If provided, please include two to five keywords, separated by the pipe symbol, e.g:
 \keywords{  Sampling distribution |  Standard error |  Normal distribution |  Quantiles |  Percentiles |  Z-scores  }  

\begin{abstract}
Because the results of random samples include an element of chance, we
can't guarantee that our inferences are correct. What we can guarantee
is that our methods usually give correct answers. We will see that the
reasoning of statistical inference rests on asking ``How often would
this method give a correct answer if I used it very many times ?'' To be
able to answer these questions we need to understand sampling
distributions and the normal curve. In this assignment you will practice
calculating quantiles and probablities from the Normal distribution. All
graphs and calculations are to be completed in an R Markdown document
using the provided template. You are free to choose any function from
any package to complete the assignment. Concise answers will be
rewarded. Be brief and to the point. Please submit both the compiled
HTML report and the source file (.Rmd) to myCourses by September 28,
2018, 11:59pm. Both HTML and .Rmd files should be saved as
`IDnumber\_LastName\_FirstName\_EPIB607\_A3'.
\end{abstract}

\dates{This version was compiled on \today}
\doi{\url{https://sahirbhatnagar.com/EPIB607/}}

\pinpfootercontents{Assignment 3 due Sepetember 28, 2018 by 11:59pm}

\begin{document}

% Optional adjustment to line up main text (after abstract) of first page with line numbers, when using both lineno and twocolumn options.
% You should only change this length when you've finalised the article contents.
\verticaladjustment{-2pt}

\maketitle
\thispagestyle{firststyle}
\ifthenelse{\boolean{shortarticle}}{\ifthenelse{\boolean{singlecolumn}}{\abscontentformatted}{\abscontent}}{}

% If your first paragraph (i.e. with the \dropcap) contains a list environment (quote, quotation, theorem, definition, enumerate, itemize...), the line after the list may have some extra indentation. If this is the case, add \parshape=0 to the end of the list environment.


\section*{Template}\label{template}
\addcontentsline{toc}{section}{Template}

The \texttt{.Rmd} template for Assignment 3 is available
\href{https://github.com/sahirbhatnagar/EPIB607/raw/master/assignments/a3/a3_template.Rmd}{here}

\section*{\texorpdfstring{The \texttt{mosaic} package
(optional)}{The mosaic package (optional)}}\label{the-mosaic-package-optional}
\addcontentsline{toc}{section}{The \texttt{mosaic} package (optional)}

The \texttt{mosaic} package provides a consistent and user-friendly
interface for descriptive statistics, plots and inference. In particular
you might find the \texttt{mosaic::xpnorm} and \texttt{mosaic::xqnorm}
functions useful for this assignment. Have a look at the
\href{https://github.com/sahirbhatnagar/EPIB607/raw/master/slides/sampling_dist/EPIB607_sampling_dist.pdf}{slides
on sampling distributions} for some examples on how to use these
functions. Remeber to install the package:

\begin{Shaded}
\begin{Highlighting}[]
\KeywordTok{install.packages}\NormalTok{(}\StringTok{"mosaic"}\NormalTok{, }\DataTypeTok{dependencies =} \OtherTok{TRUE}\NormalTok{)}
\end{Highlighting}
\end{Shaded}

Then you must load the library to access its functions:

\begin{Shaded}
\begin{Highlighting}[]
\KeywordTok{library}\NormalTok{(mosaic)}
\end{Highlighting}
\end{Shaded}

\section*{In-line R code}\label{in-line-r-code}
\addcontentsline{toc}{section}{In-line R code}

For this and future assignments you may find it useful to include
calculations from \texttt{R} directly in your text. For example, in the
following code chunk I calculate \(P(Y < 2)\) where
\(Y \sim \mathcal{N}(0,1)\), and store the result in an object called
\texttt{prob\_less\_2}:

\begin{Shaded}
\begin{Highlighting}[]
\StringTok{```}\DataTypeTok{\{r\}}
\DataTypeTok{prob_less_2 <- mosaic::xpnorm(2)}

\DataTypeTok{# round to 2 digits}
\DataTypeTok{prob_less_2 <- round(prob_less_2, 2)}
\StringTok{```}
\end{Highlighting}
\end{Shaded}

To print this result verbatim in an inline R expression use
\texttt{\textasciigrave{}r\ prob\_less\_2\textasciigrave{}} in the text.

\vspace{0.1in}

You can also call the function directly without storing the result in a
code chunk using
\texttt{\textasciigrave{}r\ round(mosaic::xpnorm(2),\ 2)\textasciigrave{}}

A sample answer would be: The probability that \(Y\) is less than 2 is
0.98.

\section{Normal probability
calculations}\label{normal-probability-calculations}

Using your method of choice, calculate the following probabilities
assuming \(Y\) is a standard normal distribution
(\(Y \sim \mathcal{N}(0,1)\)):

\begin{enumerate}
\def\labelenumi{\alph{enumi})}
\tightlist
\item
  \(P(Y < -1.80)\)
\item
  \(P(Y > -1.80)\)
\item
  \(P(Y \geq 1.60)\)
\item
  \(P(-1.8 < Y \leq 1.6)\)
\end{enumerate}

\section{HDL cholesterol}\label{hdl-cholesterol}

\begin{enumerate}
\def\labelenumi{\alph{enumi}.}
\tightlist
\item
  US women over the age of 19 have a mean (HDL) cholesterol measure of
  55 mg/dL with a standard deviation of 15.5 mg/dL. Assume HDL follows a
  Normal distribution.

  \begin{enumerate}
  \def\labelenumii{\roman{enumii}.}
  \tightlist
  \item
    What percent of women have low values of HDL, where low is defined
    to be 40 mg/dL or less?
  \item
    HDL levels of 60 mg/dL or more are believed to be protective against
    heart disease. What percent of women have protective levels of HDL?
  \item
    What proportion of women has HDL in the range of 40-60 mg/dL?
  \item
    What proportion of women has HDL in the range of 35-65 mg/dL?
  \end{enumerate}
\end{enumerate}

\section{Osteoporosis}\label{osteoporosis}

Osteoporosis is a condition in which the bones become brittle due to
loss of minerals. To diagnose osteoporosis, an elaborate apparatus
measures bone mineral density (BMD). BMD is usually reported in
standardized form. The standardization is based on a population of
healthy young adults. The Wolrd Health Organization (WHO) criterion for
osteoporosis is a BMD 2.5 or more standard deviations below the mean for
young adults. BMD measurements in a population of people who are similar
in age and sex roughly follow a standard normal distribution.

\begin{enumerate}
\def\labelenumi{\alph{enumi}.}
\tightlist
\item
  What percent of healthy young adults have osteoporosis?
\item
  Woman aged 70 to 79 are, of course, not young adults. The mean BMD in
  this age is about -2 on the standard scale for young adults. Suppose
  that the standard deviation is the same for young adults. What percent
  of this older population has osteoporosis?
\item
  Likewise, osteopenia is low BMD, defined by the WHO as a BMD between 1
  and 2.5 standard deviations below the mean of young adults. What
  percent of healthy young adults have osteopenia?
\item
  The mean BMD among women aged 70 to 79 is about -2 on the standard
  scale for young adults. Suppose that the standard deviation is the
  same as for young adults. What percent of this older population has
  osteopenia?
\end{enumerate}

\section{How deep is the ocean?}\label{how-deep-is-the-ocean}

This question is based on the
\href{https://github.com/sahirbhatnagar/EPIB607/blob/master/exercises/water/students/260194225_water_exercise_epib607.pdf}{in-class
Exercise} on sampling distributions.

\begin{enumerate}
\def\labelenumi{\alph{enumi}.}
\tightlist
\item
  For your samples of \(n=5\) and \(n=20\) of depths of the ocean,
  calculate the

  \begin{enumerate}
  \def\labelenumii{\roman{enumii}.}
  \tightlist
  \item
    sample mean (\(\bar{y}\))
  \item
    standard error of the sample mean (\(SE_{\bar{y}}\))
  \end{enumerate}
\item
  Calculate the 68\%, 95\% and 99\% confidence intervals (CI) for both
  samples of \(n=5\) and \(n=20\). You may use the following formulas to
  calculate the confidence intervals:
\end{enumerate}

\begin{longtable}[]{@{}ll@{}}
\toprule
CI & Formula\tabularnewline
\midrule
\endhead
68\% & \(\bar{y} \pm 1 \times SE_{\bar{y}}\)\tabularnewline
95\% & \(\bar{y} \pm 2 \times SE_{\bar{y}}\)\tabularnewline
99\% & \(\bar{y} \pm 3 \times SE_{\bar{y}}\)\tabularnewline
\bottomrule
\end{longtable}

Here is some sample code to calculate the CI. Suppose that based on a
sample of \(n=20\), my \(\bar{y} = 2900\) and \(SE_{\bar{y}} = 20\).
Then my confidence intervals are

\begin{Shaded}
\begin{Highlighting}[]
\NormalTok{ybar <-}\StringTok{ }\DecValTok{2900}
\NormalTok{SEybar <-}\StringTok{ }\DecValTok{20}
\CommentTok{# 68% CI}
\NormalTok{ybar }\OperatorTok{+}\StringTok{ }\KeywordTok{c}\NormalTok{(}\OperatorTok{-}\DecValTok{1}\NormalTok{,}\DecValTok{1}\NormalTok{) }\OperatorTok{*}\StringTok{ }\NormalTok{SEybar}
\CommentTok{# 95% CI}
\NormalTok{ybar }\OperatorTok{+}\StringTok{ }\KeywordTok{c}\NormalTok{(}\OperatorTok{-}\DecValTok{2}\NormalTok{,}\DecValTok{2}\NormalTok{) }\OperatorTok{*}\StringTok{ }\NormalTok{SEybar}
\CommentTok{# 99% CI}
\NormalTok{ybar }\OperatorTok{+}\StringTok{ }\KeywordTok{c}\NormalTok{(}\OperatorTok{-}\DecValTok{3}\NormalTok{,}\DecValTok{3}\NormalTok{) }\OperatorTok{*}\StringTok{ }\NormalTok{SEybar}
\end{Highlighting}
\end{Shaded}

Note that I have provided this code in the template as well. Take a look
at the template before starting these calculations.

\begin{enumerate}
\def\labelenumi{\alph{enumi}.}
\setcounter{enumi}{2}
\tightlist
\item
  What do you notice about the size of the three intervals for a given
  sample size?
\end{enumerate}

%\showmatmethods


\bibliography{pinp}
\bibliographystyle{jss}



\end{document}

